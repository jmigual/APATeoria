% !TEX root = ../main.tex

\section{Sequential processes}

\subsection{Remarks on Project}
\begin{itemize}
  \item Good requirements are SMART (Specific, Measurable, Attainable, Realistic, Time-bound)
  \begin{itemize}
    \item Read Measurable as verifiable
    \item Separate design from requirements
    \item Time bound means that you can implement it in for the time set. Not applicable for this
    project
  \end{itemize}
  \item Abstract from timing
  \item Formulate requirements about things you can control
  \item Do not formulate requirements procedurally
  \begin{itemize}
    \item Don't: when $R_1$ picks up a wafer from the input stack, it is placed in the airlock,
    and placed under $L$ by robot 2
    \item Do: $R_i$ should not place a wafer onto position $P$ that already contains a wafer,
    for all $P$ other than the output stacks.
  \end{itemize}
  \item Descriptions of sequences of actions are not needed
  \item For this project, the user input is not clear. There's no need to model it
\end{itemize}

\subsection{Goal}

\begin{itemize}
  \item Describe sequential processes using process algebra
  \item Reason about sequential processes
  \item Prove processes are equivalent
  \item Draw LTS for given processes
\end{itemize}

\subsection{Recap}

\begin{itemize}
  \item $\delta$ deadlock
  \item $a \cdot b$ Where $a$ moves to a new state and $b$ to a new one
  \item $a + b$ Where we can choose between two different states
\end{itemize}

This is basically all that you need to model sequential processes, although we need a bit more
things like:
\begin{itemize}
  \item How can we repeat?
  \item How can we add data to our model?
\end{itemize}

\subsection{Different process terms represent bisimilar processes}
$a \cdot b + a \cdot b$ and $a \cdot b$

\begin{figure}[H]
  \centering
  \begin{tikzpicture}
    \node[state](a) {};
    \node[state, below left = of a] (aa) {};
    \node[state, below right = of a] (ab) {};
    \node[state, below = of aa] (aac) {};
    \node[state, below = of ab] (abc) {};

    \draw
      (a) edge node[above left] {$a$} (aa)
      (a) edge node[above right] {$a$} (ab)
      (aa) edge node[left] {$b$} (aac)
      (ab) edge node[right] {$b$} (abc)
      ;
  \end{tikzpicture}
\end{figure}

\subsection{Axioms}

\begin{itemize}
  \item A1 $x + y = y + x$ Commutativity
  \item A2 $(x + y) + z = x + (y + z)$ Associativity of $+$
  \item A3 $x + x = x$ Idempotence
  \item A4 $(x + y) \cdot z = x \cdot z + y \cdot z$ Right distributivity
  \item A5 $x \cdot (y \cdot z) = (x \cdot y) \cdot z$ Associativity of $\cdot$
\end{itemize}

\textbf{Theorem} These axioms characterise exactly strong bisimulation for processes expressible
with actions, $+$ and $\cdot$.

\subsection{A derivation}

\begin{align*}
  &(a + b) \cdot c \cdot (d + d) \overset{?}{=} (b \cdot + a \cdot c) \cdot d \\
  &(a + b) \cdot c \cdot (d + d) \overset{A3}{=} \\
  &(a + b) \cdot c \cdot d \overset{A5}{=} \\
  &((a + b) \cdot c) \cdot d \overset{A4}{=} \\
  &(a \cdot c + b \cdot c) \cdot d \overset{A1}{=} \\
  &(b \cdot c + a \cdot c) \cdot d
\end{align*}

\subsection{Be aware: Left distribution}

Left distribution: $x \cdot (y + z) = x \cdot y + x \cdot z$

Invalid in strong bisimulation

Valid for trace equivalence

\textbf{Theorem} Axioms A1-A5 $+$ left distribution is sound and complete for trace equivalence

\subsection{Axioms for $\delta$}

\begin{itemize}
  \item A6: $x + \delta = x$
  \item A7: $\delta \cdot x = \delta$
\end{itemize}

\subsection{Exercise I}

\begin{align*}
  & (a \cdot b + a \cdot \delta) \cdot (\delta + a) \overset{?}{=} (a \cdot \delta + a)\cdot b \cdot a \\
  (a \cdot b + a \cdot \delta) \cdot (\delta + a) \overset{A6}{=} \\
  (a \cdot b + a \cdot \delta) \cdot a \overset{A7}{=} \\
  (a \cdot b + a \cdot \delta \cdot b) \cdot a \overset{A4}{=} \\
  (a + a \cdot \delta) \cdot b \cdot a \overset{A1}{=} \\
  (a \cdot \delta + a) \cdot b \cdot a
\end{align*}

\subsection{Exercise II}
Why is it not possible to prove $a \cdot (b + c) = a \cdot b + a \cdot c$ using the axioms for
strong bisimulation.

The identity is not valid in bisimulation and the axioms are sound

\subsection{Exercise III}
Assume that we know for arbitrary processes $x$ and $y$ that $x + y = \delta$. Prove that $x = \delta$.

\subsection{Recursive behaviour cannot yet be expressed}
We somehow need to express this behaviour.

\subsection{Recursive process specification}

$X$ = 'process expression'

Process variable

\begin{align*}
  & X = a \cdot X \\
  & Y = a \cdot b + c \cdot d \cdot Y \\
  & U = a \cdot W, W = b \cdot U
\end{align*}

\subsection{Guarded recursive process equation}

If process variables are preceded by an action, then process variable is called \emph{guarded}

\begin{align*}
  & a \cdot X \\
  & (a + b) \cdot X + b \cdot Y \\
  & X + a \cdot b \cdot Y \quad (X \text{ is \emph{not} guarded})
\end{align*}

If all process variables are guarded, the recursive specification is \emph{guarded}

\begin{align*}
  & X = a \cdot X \\
  & X = a \cdot X + (b \cdot c + d) \cdot X \\
  & Y = Y + a \cdot b \cdot Y \quad (\text{\emph{not} guarded})
\end{align*}

\subsection{Axioms for the conditional operator}

\begin{align*}
  & true \rightarrow x \lozenge y = x \\
  & false \rightarrow x \lozenge y = y
\end{align*}

\subsection{Axioms for the sum operator}

\begin{itemize}
  \item SUM1 $\sum_{d_:D} x = x$
  \item SUM3 $\sum_{d:D} X(d) = X(e) + \sum_{d:D} X(d)$
  \item SUM4 $\sum_{d:D} (X(d) + Y(d)) = \sum_{d:D} X(d) + \sum_{d:DY} (d)$
  \item SUM5 $(\sum_{d:D} X(d)) \cdot y \sum_{d:D} X(d) \cdot y$
\end{itemize}



















