% !TEX root = ../main.tex

\section{Specifying LTSs and applying behavioural equivalences}

\textbf{LTS}: Label Transition Systems

How can we actually use some of the tools for the project?


\textbf{Exercise}: Given the label transition system

\begin{figure}[H]
  \centering
  \begin{tikzpicture}
    \node[state] (aa) {};
    \node[state, below = of aa] (ba) {};
    \node[state, below = of ba] (ca) {};
    \node[state, right = of aa] (ab) {};
    \node[state, below = of ab] (bb) {};
    \node[state, below = of bb] (cb) {};
    \node[state, right = of ab] (ac) {};
    \node[state, below = of ac] (bc) {};
    \node[state, below = of bc] (cc) {};

    \draw
      (aa)    edge node[above] {b}    (ab)
              edge node[right] {a}    (ba)
      (ba)    edge node[above] {b}    (bb)
              edge node[right] {a}    (ca)
      (ca)    edge node[above] {b}    (cb)
              edge[bend left] node[left] {a} (aa)
      (ab)    edge node[above] {b}    (ac)
              edge node[right] {a}    (bb)
      (bb)    edge node[above] {b}    (bc)
              edge node[right] {a}    (cb)
      (cb)    edge node[above] {b}    (cc)
              edge[bend left] node[left, near end] {a} (ab)
      (ac)    edge node[right] {a}    (bc)
      (bc)    edge node[right] {a}    (cc)
      (cc)    edge[bend right] node[right] {a} (ac);
  \end{tikzpicture}
\end{figure}

\textbf{Exercise}: Slide 4, are these LTSs branching bisimilar?

We can relate the $\tau$ transition by doing nothing int the left LTS

\subsection{Goals}
\begin{itemize}
  \item Model small LTSs in mCRL2
  \item Explain what specification means
  \item Generate LTS from specification
  \item Apply behavioural equivalences to small examples
\end{itemize}

\subsection{Structure of the mCRL2 toolset}
The input is written as process algebra, something similar to a programming language to 
describe processes. Then it's linearized and linear process algebra is created, similar to an
assembly language. Finally a LTS is generated.

When specifying a system in mCRL2 the starting point are the actions. An action is, in fact, a 
LTS.

\subsubsection{Multi-actions}

Actions happening simultaneously

\begin{align*}
  & a | b \\
  & a | b | c \\
  & a | a \\
  & a | a | a |a  \\
  & \tau
\end{align*}

\subsubsection{Exercise II}

Are the following pairs of processes strongly bisimilar?

\begin{itemize}
  \item $a \cdot (b + c)$ and $a \cdot b + a \cdot c$. 

  No, these are not strongly bisimilar
  \item $a \cdot b + a \cdot b$ and $a \cdot b$
  \item $a \cdot b + a \cdot c$ and $(a + a) \cdot (b + c)$

  No, this is the same as the first case

\end{itemize}

\subsection{Transition systems with data}

Actions:
\begin{itemize}
  \item \emph{get(d)}
  \item \emph{deliver(d)}
\end{itemize}

$$
D = \{d_1, d_2\}
$$

\subsubsection{Alternating bit protocol}

When you have an unreliable transfer channel (e.g. internet)

You can get:
\begin{itemize}
  \item A successful transmission
  \item An unsuccessful transmission
\end{itemize}

You send a bit and expect an acknowledgement bit.

\subsection{Summary}
\begin{itemize}
  \item Modeled LTSs in mCRL2
  \item We generate LTSs from a specification
  \item We applied behavioural equivalences
  \item We saw how we can use tools and techniques to analyses a real protocol
  \item Tools used: mcrl2-gui, mcrl2ide, mcrl2xi, mcrl22lps, lps2lts, ltsconvert, ltscompare
\end{itemize}


