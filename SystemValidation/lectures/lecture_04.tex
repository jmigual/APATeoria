% !TEX root = ../main.tex

\section{Abstract data types}

To model a model we need to be able to model the data too. Similar to functional programming.

\subsection{Goal}
\begin{itemize}
  \item Describe abstract data types (ADTs)
  \item Be able to use ADTs in mCRL2
\end{itemize}

\subsection{Declaration of data type}

\textbf{sort} D;

It represents a non-empty domain. $D$ can have infinite elements.

The most basic data type are the booleans.

\textbf{sort} $\mathbb{B}$;

\textbf{cons} \emph{true}, \emph{false}: $\mathbb{B}$

Special semantic rule: The elements representing \emph{true} and \emph{false} of sort $\mathbb{B}$
(\emph{Bool}) must be different.

\begin{align*}
  \text{\textbf{sort} } & \mathbb{B}; \\
  \text{\textbf{cons} } & true, false: \mathbb{B}; \\
  \text{\textbf{map} } & not: \mathbb{B} \Rightarrow \mathbb{B}; \\
  & and, or: \mathbb{B} \times \mathbb{B} \Rightarrow \mathbb{B}; \\
  \text{\textbf{var} } & b: \mathbb{B} \\
  \text{\textbf{eqn} } & not(true) = false; \\
  & not(false) = true; \\
  & not(not(true)) = true; \\
  \text{\textbf{eqn} } & and(true, b) = b; \\
  & and(false, b) = false; \\
  & and(b, true) = b; \\
  & and(b, false) = false; \\
  \text{\textbf{eqn}} & or(true, b) = true; \\
  & or(false, b) = b; \\
  & or(b, true) = true; \\
  & or(b, false) = b; \\
\end{align*}

\subsubsection{Induction/case distinction on Bool}
Can we prove $and(b, b) = b$ ?

As true and false are the constructors of $\mathbb{B}$ we only have to prove that 
$and(true, true) = true$ and that $and(false, false) = false$.

Prove that $and(b, c) = and(c, b)$:

\begin{align*}
  \text{case } & b = true \\
  & and(true, c) = c \\
  & \text{case } c = true \\
  & & and(true, true) = and(true, true) \\
  & \text{case } c = false \\
  & & \underbrace{and(true, false)}_{false} = \underbrace{and(false, true)}_{false} \\
\end{align*}

\subsection{Specification numbers. Peano arithmetic}

\begin{align*}
  \text{\textbf{sort} } & Nat; \\
  \text{\textbf{cons} } & zero: Nat; \\
  & succ: Nat \rightarrow Nat; \\
  \text{\textbf{map }} & add: Nat \times Nat \rightarrow; \\
  \text{\textbf{var }} & n, m: Nat; \\
  \text{\textbf{eqn }} & add(zero, zero) = zero; \\
  & add(zero, succ(n)) = succ(n); \\
  & add(succ(n), zero) = succ(n); \\
  & add(succ(n), succ(m)) = succ(add(n, succ(m))); \\
  \text{\textbf{map }} & <: Nat \times Nat \rightarrow \mathbb{B}; \\
  \text{\textbf{eqn }} & n < zero = false; \\
  & zero < succ(n) = true; \\
  & succ(n) < succ(m) = n < m;
\end{align*}

\begin{align*}
  & zero & 0 \\
  & succ(zero) & 1 \\
  & succ(succ(zero)) & 2 \\
\end{align*}

\subsubsection{Induction on Natural numbers}
Prove a property of \emph{zero}

Prove a property for $succ(n)$, assuming that it has been proven for $n$.

Then the property has been prover for all natural numbers $n$.

\subsection{Specification of efficient natural numbers}

Peano arithmetic is too inefficient for system validation.

\begin{align*}
  \text{sort } & \mathbb{N}^+; \qquad \text{sort } Pos;
\end{align*}

No maximum number

\subsection{Natural numbers}

\begin{align*}
  \text{sort } & \mathbb{N}; \qquad \text{sort} Nat; \\
  \text{cons } & @c0: \mathbb{N}; \\
  & @cNat: \mathbb{N}^+ \rightarrow \mathbb{N}; \\
  \text{map } & \simeq: \mathbb{N} \times \mathbb{N} \rightarrow \mathbb{N}^+
\end{align*}







