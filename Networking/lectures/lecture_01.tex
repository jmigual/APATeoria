% !TEX root = ../main.tex

\section{Introduction}

\begin{itemize}
  \item All information is on Brightspace
  \item Study material $\rightarrow$ slides
  \item Bonus point
  \begin{itemize}
    \item Nice, working examples that can be used in networking
    \item Solution to open problems, solutions that may not be yet solved
  \end{itemize}
  \item Books, you are not obliged to buy the books (they are expensive)
  \begin{itemize}
    \item Performance Analysis of Complex Networks and Systems
  \end{itemize}
  \item Additional books
  \begin{itemize}
    \item Network Science
    \item Networks: An introduction
  \end{itemize}
\end{itemize}

\subsection{Everything on the graph}

Example: The brain network is amazing
\begin{itemize}
  \item $N=10^{10}$ neurons
  \item $L=10^{14}$ connections
  \item 500.000 km of wiring
  \item Each neuron has an average connection to 10.000 neurons
\end{itemize}

We see networks everywhere

\subsection{Understanding Network Function \& Graph}

Designing robust infrastructures: Internet, electricity grid...

\begin{itemize}
  \item Are there properties common to all complex networks?
  \item if so, why?
  \item Can we formulate a general theory of the structure, evolution and dynamics of complex networks?
  \item How do complex networks give rise to adaptive, living, intelligent behaviour?
  \item How can we learn from nature to design robust, efficient, self-adaptive man-made networks?
\end{itemize}

\subsubsection{A new discipline}

\begin{itemize}
  \item Networks are not new, why is NS only one decade old?
  \item Networks where a given constant
  \item Mid 90s, the internet and www raised questions about their topology
  \item We were building network infrastructures without knowing how these truly worked
  \item Network design was more pushed by physicists than by engineers
\end{itemize}

\subsubsection{Characteristics}

\begin{itemize}
  \item Interdisciplinary nature
  \item Empirical, data-driven
  \item Mathematical nature
  \item Computational nature. Algorithms
\end{itemize}

\subsection{Representation of networks}

A network always consists in two separate layers:
\begin{itemize}
  \item Service (function)
  \item Topology (graph)
\end{itemize}

Example Microsoft vs Apple:
\begin{itemize}
  \item Microsoft decided to center in functionality so they decided to write they operating
  system for all the different devices
  \item Apple decided to build their own hardware, it's less flexible but more reliable and secure
\end{itemize}

Making services is way more complex than the graph itself. Problems in graph theory can be 
explained in very low detail, while on the service problems are more complex ``I would like
that my service is more robust''. 

\subsection{What is a graph?}

A graph $\mathcal{G}$ consists on a set of $N$ nodes (vertices) that are connected by a set of 
$L$ links (edges).

\subsubsection{Birth of graph theory: the Königsberg bridge problem (Euler, 1736)}

Can one walk across the seven bridges and never traverse the same bridge twice?

\begin{figure}[H]
  \centering
  \begin{tikzpicture}
    \node[state] (A) {A};
    \node[state, above right = of A] (C) {C};
    \node[state, below right = of A] (B) {B};
    \node[state, right = 2 of A] (D) {D};

    \draw[-]   (A)   edge    (D)
                  edge[bend right]    (B)
                  edge[bend left]    (B)
                  edge[bend right]    (C)
                  edge[bend left]    (C)
            (B)   edge[bend right]    (D)
            (C)   edge[bend right]    (D);
  \end{tikzpicture}
\end{figure}

An Eulerian walk exists if all nodes have even degree, only zero or two nodes with odd degree





