% !TEX root = ../main.tex

\section{Introduction}

\begin{itemize}
  \item All information is on Brightspace
  \item Study material $\rightarrow$ slides
  \item Bonus point
  \begin{itemize}
    \item Nice, working examples that can be used in networking
    \item Solution to open problems, solutions that may not be yet solved
  \end{itemize}
  \item Books, you are not obliged to buy the books (they are expensive)
  \begin{itemize}
    \item Performance Analysis of Complex Networks and Systems
  \end{itemize}
  \item Additional books
  \begin{itemize}
    \item Network Science
    \item Networks: An introduction
  \end{itemize}
\end{itemize}

\subsection{Everything on the graph}

Example: The brain network is amazing
\begin{itemize}
  \item $N=10^{10}$ neurons
  \item $L=10^{14}$ connections
  \item 500.000 km of wiring
  \item Each neuron has an average connection to 10.000 neurons
\end{itemize}

We see networks everywhere

\subsection{Understanding Network Function \& Graph}

Designing robust infrastructures: Internet, electricity grid...

\begin{itemize}
  \item Are there properties common to all complex networks?
  \item if so, why?
  \item Can we formulate a general theory of the structure, evolution and dynamics of complex networks?
  \item How do complex networks give rise to adaptive, living, intelligent behaviour?
  \item How can we learn from nature to design robust, efficient, self-adaptive man-made networks?
\end{itemize}

\subsubsection{A new discipline}

\begin{itemize}
  \item Networks are not new, why is NS only one decade old?
  \item Networks where a given constant
  \item Mid 90s, the internet and www raised questions about their topology
  \item We were building network infrastructures without knowing how these truly worked
  \item Network design was more pushed by physicists than by engineers
\end{itemize}

\subsubsection{Characteristics}

\begin{itemize}
  \item Interdisciplinary nature
  \item Empirical, data-driven
  \item Mathematical nature
  \item Computational nature. Algorithms
\end{itemize}

\subsection{Representation of networks}

A network always consists in two separate layers:
\begin{itemize}
  \item Service (function)
  \item Topology (graph)
\end{itemize}

Example Microsoft vs Apple:
\begin{itemize}
  \item Microsoft decided to center in functionality so they decided to write they operating
  system for all the different devices
  \item Apple decided to build their own hardware, it's less flexible but more reliable and secure
\end{itemize}

Making services is way more complex than the graph itself. Problems in graph theory can be 
explained in very low detail, while on the service problems are more complex ``I would like
that my service is more robust''. 

\subsection{What is a graph?}

A graph $\mathcal{G}$ consists on a set of $N$ nodes (vertices) that are connected by a set of 
$L$ links (edges).

\subsubsection{Birth of graph theory: the Königsberg bridge problem (Euler, 1736)}

Can one walk across the seven bridges and never traverse the same bridge twice?

\begin{figure}[H]
  \centering
  \begin{tikzpicture}
    \node[state] (A) {A};
    \node[state, above right = of A] (C) {C};
    \node[state, below right = of A] (B) {B};
    \node[state, right = 2 of A] (D) {D};

    \draw[-]   (A)   edge    (D)
                  edge[bend right]    (B)
                  edge[bend left]    (B)
                  edge[bend right]    (C)
                  edge[bend left]    (C)
            (B)   edge[bend right]    (D)
            (C)   edge[bend right]    (D);
  \end{tikzpicture}
\end{figure}

An Eulerian walk exists if all nodes have even degree, only zero or two nodes with odd degree

\subsection{Three equivalent representations of a graph}

\begin{itemize}
  \item Topology domain
  \item Spectral domain
  \item Geometric domain: You can represent each graph in geometric space and talk about
  them like having volumes, this way you can find a geometric view of the graph in a 
  $(N-1)$-dimensional space.
  
\end{itemize}

\subsubsection{Topology domain}

\begin{itemize}
  \item List of neighbors: for each node, the set of direct neighbors listed
  \item List of links
  \item Adjacency matrix
  \begin{itemize}
    \item If there is a link between node $i$ and $j$, then $a_{ij} = 1$, else $a_{ij} = 0$.
    \item Number of neighbors of node $i$ i s the degree: $d_i = \sum_{k = 1}^{N} a_{ik}$
    \item
    $
    d = A \cdot u =
    \begin{bmatrix}
      1 \\ 1 \\ \vdots \\ 1
    \end{bmatrix}_{N \times 1}
    $
  \end{itemize}
  \item Incidence matrix $B_{N \times L}$
  \begin{itemize}
    \item Col sum B is zero: $u^T B = 0$
  \end{itemize}
  \item Laplacian matrix $Q_{N \times N} = B_{N \times L} \cdot (B^T)_{L \times N} = \Delta - A$
  \begin{itemize}
    \item Basic property: $Q \cdot u = 0$
    \item $Q \cdot u = B \cdot B^T \cdot u = 0$ because $0 = u^T \cdot B = B^T u$
    \item $Q \cdot z = \mu \cdot z$
    \item $u$ is an eigenvector of $Q$ belonging to eigenvalue $\mu = 0$
    \item Since $BB^T$ is symmetric, so are $A$ and $Q$. Although $B$ specifies directions,
    $A$ and $Q$ lost this info here.
  \end{itemize}
\end{itemize}

\subsubsection{Link weights}
\begin{itemize}
  \item \textbf{Unweighted graph}
  \item \textbf{Weighted graph}
  \begin{itemize}
    \item This nodal heterogeneity can be specified by a node weight vector $v \ne u$.
    \item A weighted adjacency matrix $W$ specifies the link weights
  \end{itemize}
\end{itemize}

\subsubsection{Classes of networks}
\begin{itemize}
  \item Trees, $L- 1$. Only connected graph with less links than nodes
  \item Ring $C_N$, $L=N$
  \item Complete graph $K_N$. $L=\frac{N(N-1)}{2}$. Degree of every node is $N-1$.
\end{itemize}

\subsubsection{Classes of grpahs}
\begin{itemize}
  \item Regular graph
  \item Lattice, very used in physics
  \item Special type of trees:
  \begin{itemize}
    \item Star, one central node and all the other ones connected to the central one
    \item Path graph, every node connects to the next one creating a path
  \end{itemize}
  \item A zoo of special graphs (Caley, Petersen, Krautz, Ramanuyan...)
  \item Classes of random graphs
  \item Null graph, defined by $A = 0$, is a set of $N$ nodes without any link
  \item Complement $G^C$ of a grpah $G$
  \begin{itemize}
    \item If there is a link between two nodes in $G$, there is none in $G^C$
    \item $A^C = J - I - A$ (where $J$ is the all one matrix)
  \end{itemize}
  \item Line graph $I(G)$ of a graph $G$:
  \begin{itemize}
    \item Each link in $G$ is a node in $I(G)$ and two nodes in $I(G)$ are connected by a link
    if the corresponding two links in $G$ have a node in common.
  \end{itemize}
\end{itemize}

\subsubsection{Subgraph of a graph \label{sec:subgraph}}

You can divide the graph by extracting a subgraph and swapping some columns in the adjacency matrix
then the submatrices are related to the subgraph and the complementary one. The other two 
submatrices show the relations between the two subgraphs

$$
\begin{pmatrix}
  A_S & B \\
  C & A_{G \backslash S}
\end{pmatrix}
$$

The interconnection matrix $C = B^T$ is not a square matrix.

\subsubsection{Interdependent Networks}
\begin{itemize}
  \item Single network
  \begin{itemize}
    \item Transport over links of one type
    \item Electricity network, the Internet, water supply network, Telephony network...
  \end{itemize}
  \item Interdependent network
  \begin{itemize}
    \item Most networks do not operate in isolation
    \item A telecommunication network can only function when there is electricity
  \end{itemize}
\end{itemize}

You can represent it as a layer of multiple networks and show the dependence links of one network
with another, this is related to \autoref{sec:subgraph}. The major difficulty is how to determine
the interconnection pattern.

\subsubsection{Some fascinations}
\begin{itemize}
  \item Szemeredi's regularity lemma: every graph can be partitioned in about equal parts,
  with most links connecting different parts and the number of links between any two parts is about
  uniformly distributed.
  \item If every pair in a population has exactly one friend in common, then there must be
\end{itemize}






