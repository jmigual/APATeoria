% !TEX root = ../main.tex

\section{Behavioural equivalences}

\subsection{When transition systems are equal?}

\begin{figure}[H]
  \centering
  \begin{subfigure}[b]{.4\textwidth}
    \centering
    \begin{tikzpicture}
      \node[state] (0) {};
      \node[state, below = of 0] (1) {};
      \node[state, below left = of 1] (2) {};
      \node[state, below right = of 1] (3) {};

      \draw (0)   edge    node {a}  (1)
            (1)   edge    node {b}  (2)
                  edge    node {c}  (3);
    \end{tikzpicture}
    \caption{System a \label{fig:02_system_a}}
  \end{subfigure}
  ~
  \begin{subfigure}[b]{.4\textwidth}
    \centering
    \begin{tikzpicture}
      \node[state] (0) {};
      \node[state, below right = of 0] (1) {};
      \node[state, below left = of 0] (2) {};
      \node[state, below = of 1] (3) {};
      \node[state, below = of 2] (4) {};

      \draw (0)   edge    node {a}  (1)
                  edge    node {a}  (2)
            (1)   edge    node {b}  (3)
            (2)   edge    node {c}  (4);
    \end{tikzpicture}
    \caption{System a \label{fig:02_system_b}}
  \end{subfigure}
  \caption{Are these two systems equivalent?}
\end{figure}



Usually it depends in the interaction on when two systems are equal. 
Not \emph{bisimulation equivalent}.


\subsection{Bisimulation equivalence}

We say that a relation \( \pmb{R} \subseteq S \times S \) is a bisimulation relation iff for all states
\( s,t \in S \) it holds that if \( sRt \).

\begin{itemize}
  \item if \( s \xrightarrow{a} s' \) for some \( s'\in S\), then there is some \( t' \in S \)
  such that \( t \xrightarrow{a} t' \) and \( s'\pmb{R}t' \).
  \item if \( t \xrightarrow{a} t' \) for some \( t'\in S\), then there is some \( s' \in S \)
  such that \( s \xrightarrow{a} t' \) and \( s'\pmb{R}t' \).
  \item \( s \in T \) iff \( t \in T \)
\end{itemize}

Two states \( s,t \in S \) are called bisimilar 


\textbf{Theorem} Every transition system has a \emph{unique} minimal transition system that is
bisimulation equivalent to it.

\begin{figure}[H]
  \centering
  \begin{subfigure}[b]{.45\textwidth}
    \centering
    \begin{tikzpicture}
      \node[state] (0) {};
      \node[state, below left = of 0] (1) {};
      \node[state, below right = of 0] (2) {};
      \node[state, below = of 1] (3) {};
      \node[state, below = of 2] (4) {};

      \draw (0)   edge    node {a}    (1)
            (1)   edge    node {a}    (3)
            (3)   edge    node {a}    (4)
            (4)   edge    node {a}    (2)
            (2)   edge    node {a}    (0);
    \end{tikzpicture}
    \caption{System a}
  \end{subfigure}
  \centering
  \begin{subfigure}[b]{.45\textwidth}
    \begin{tikzpicture}
      \node[state] (0) {};
      
      \draw (0) edge[loop below] node {a} (0);
    \end{tikzpicture}
    \caption{System b}
  \end{subfigure}
  \caption{These two systems have a bisimulation relation}
\end{figure}

\subsection{Trace equivalence}


\begin{figure}[H]
  \begin{subfigure}[b]{.4\textwidth}
    \centering
    \begin{tikzpicture}
      \node[state] (0) {};
      \node[state, below = of 0] (1) {};
      \node[state, below left = of 1] (2) {};
      \node[state, below right = of 1] (3) {};

      \draw (0)   edge    node {a}  (1)
            (1)   edge    node {b}  (2)
                  edge    node {c}  (3);
    \end{tikzpicture}
    \caption{System a, trace is \( \{ \epsilon, a, a \cdot b, a \cdot c \} \)}
  \end{subfigure}
  ~
  \begin{subfigure}[b]{.4\textwidth}
    \centering
    \begin{tikzpicture}
      \node[state] (0) {};
      \node[state, below right = of 0] (1) {};
      \node[state, below left = of 0] (2) {};
      \node[state, below = of 1] (3) {};
      \node[state, below = of 2] (4) {};

      \draw (0)   edge    node {a}  (1)
                  edge    node {a}  (2)
            (1)   edge    node {b}  (3)
            (2)   edge    node {c}  (4);
    \end{tikzpicture}
    \caption{System b, trace is \( \{ \epsilon, a, a \cdot b, a \cdot c \} \)}
  \end{subfigure}

  \caption{These two systems are not bisimilar but have a trace equivalence}
\end{figure}

For the trace equivalence all the partial traces are included. Trace equivalence does not 
preserve deadlocks. To verify a system, never use trace equivalence because it does not counts 
the deadlocks.

Calculating bisimulation on a finite transition system with \( n \) states and \( m \) transitions
requires \emph{insert formula here}

\begin{displayquote}
For deterministic transition systems bisimulation and trace equivalence coincide
\end{displayquote}

\subsection{Useful facts}

If you show that two transitions systems are not trace equivalent you could find one trace in 
one but not in the other.

Exercise from 20: Are these two trace equivalent? 

Yes these two are trace equivalent because we always have 3a followed by another letter. 

Exercise from 21: Are these two bisimilar?

No, they are not bisimilar because you can end in a transition where a b transition must be mimicked
by another b transition but this is not possible.


\subsection{Branching bisimulation}

A relation \(\pmb{R}\) is a \emph{branching bisimulation} relation iff, some branches can be 
skipped 

\begin{figure}[H]
  \centering
  \begin{tikzpicture}
    \node[state] (a0) {};
    \node[state, below = of a0] (a1) {};
    \node[state, below = of a1] (a2) {};
    \node[state, right = of a0] (b0) {};
    \node[state, below = of b0] (b1) {};
    \node[state, below = of b1] (b2) {};
    \node[state, below = of b2] (b3) {};

    \draw (a0)    edge    node {a}    (a1)
          (a1)    edge    node {b}    (a2)
          (b0)    edge    node {a}    (b1)
          (b1)    edge    node {\(\tau\)}    (b2)
          (b2)    edge    node {b}    (b3);
  \end{tikzpicture}

  \caption{These two systems are branching bisimilar}
\end{figure}

\subsection{Combining behaviour}

If we combine branching bisimilar systems we do not obtain branching bisimilar systems.

\subsection{Rooted branching bisimulation}

A rooted branching bisimulation relation \( \pmb{R}\) is a branching bisimulation that also
satisfies that the first transition is the same, without skips.

Strong bisimulation implies rooted branching bisimulation implies branching bisimulation.

\subsection{When to use which behavioural equivalence?}

\begin{itemize}
  \item Strong bisimulation, always safe
  \item Trace equivalence, 
  \item Divergence preserving branching bisimulation. Used to remove internal steps. Always safe choice.
  \item Branching bisimulation. Safe choice when eventualities are not important, as it removes
  \(\tau\)-loops.
  \item Weak bisimulation. For practical purposes generally equivalent to branching bisimulation.
  \item Weak trace equivalence. Removes all \(\tau\)s. Does not preserve branching behaviour.
  Use with care.
\end{itemize}

Exercice I (pag. 43): We have a complex system where we hide many actions. We want to see whether
the behaviour is free from deadlocks. What should we use?

1) Strong bisimulation. It is always safe but we hide many actions, although correct it is not 
the best scenario
2) Divergence preserve branching bisimulation. Is is always safe, seems like the best choice
3) Branching bisimulation. In this case it is the safe choice, any deadlock that exists is going
to be preserved after branching bisimulation reduction.
4) Trace equivalence. No, it does not preserve deadlocks
5) It does not preserve deadlocks neither.

\subsection{Summary}

We defined the following behavioural equivalences:
\begin{itemize}
  \item (Weak) trace equivalence
  \item Strong bisimulation
  \item (Rooted) (Divergence preserving) branching bisimulation
\end{itemize}

We saw some guidelines on when to use which behavioural equivalence



