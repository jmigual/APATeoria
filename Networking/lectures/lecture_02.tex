% !TEX root = ../main.tex

\section{Graph metrics}

What is essential on the graph?

Types of properties:
\begin{enumerate}
  \item Local: property of the surrounding node
  \item Global: characteristic of the entire graph
\end{enumerate}

\subsection{Graph metrics}
\subsubsection{Degree}

Degree $d_j$ of node $j$: number of neighbours of $j$

\begin{figure}
  \centering
  \begin{tikzpicture}
    \node[state] (1) {1};
    \node[state, right = 2 of 1] (3) {3};
    \node[state, below right = of 1] (2) {2};
    \node[state, below right = of 3] (4) {4};
    \node[state, below = 2 of 1] (6) {6};
    \node[state, below = 2 of 3] (5) {5};

    \draw[-]    (1)   edge    (3)
                      edge[bend right]    (2)
                      edge    (6)
                (2)   edge[bend left]     (6)
                      edge[bend left]     (5)
                      edge    (3)
                (5)   edge    (4)
                      edge[bend left]     (6)
                (4)   edge    (3);
                    
  \end{tikzpicture}
\end{figure}

$$
\sum_{j=1}^N d_j = 2L
$$

Average degree in $\mathcal{G}$ equals:
$$
E[D] = \frac{1}{N} \sum_{j=1}^N d_j = \frac{2L}{N}
$$

Bounds (connected graph):
$$
2 - \frac{2}{N} \le E[D] \le N-1
$$

$$
L(K_N) = 
\begin{pmatrix}
  N \\ 2
\end{pmatrix} = \frac{N (N-1)}{2}
$$

Degree vector (row sum A): $d = Au$
\begin{itemize}
  \item Coordinate representation of the vector: $u = (1, 1, ..., 1)$
  \item Matrix representation of the vector $u^T = [1 1 ... 1]$. A vector in $m$ dimensions is
  a $m \times 1$ matrix (elements in a column)
\end{itemize}

\emph{Basic law of the degree}:
$$
u^T d = 2L \xRightarrow{thus} u^Td = u^T Au = 2L
$$

\begin{itemize}
  \item At least two nodes in G have the same degree
  \begin{itemize}
    \item A degree $N-1$ node excludes the existence of a zero degree node
  \end{itemize}
  \item The number of nodes with \textbf{odd} degree is \textbf{even}
\end{itemize}

You can also create a histogram plotting all the degrees. This allows you to have the average
and the variance of the degrees (slide 6).

Example: degree of an airport
The probability is almost linear in a log scale. It has been found that many real life problems
have an almost linear degree distribution.
$$
Pr[D_{\text{Air}} = k] \sim k^{-1.21}
$$

Internet:
$$
Pr[D_{\text{Internet}} = k] \sim k^{-\tau}, \quad \tau \in (2.2, 2.5)
$$

Many networks have a power law degree distribution (scale-free networks)

\subsubsection{Clustering coefficient}
The degree can be seen as a global metric. This one goes a bit step further.

The clustering coefficient of node $v$ is 
$$
c_G(v) = \frac{2y}{d_v(d_v - 1)} = \frac{y}{y_{max}}
$$

where $y$ is the number of links between neighbors and 
$
y_{max} = \begin{pmatrix}
  d_v \\ 2
\end{pmatrix}
$

This number tells you how connected your environment is. The clustering coefficient of a graph G:
$$
c_G = \frac{1}{N} \sum_{v = 1}^N c_G(v)
$$

\subsubsection{Adjacency matrix A and walks}
\begin{itemize}
  \item Walk of length $k$ from node $i$ to $j$: succession of $k$ links (arcs)
  $n_0 \rightarrow n_1)(n_1 \rightarrow n_2) ... (n_{k-1} \rightarrow n_k)$ where $n_0 = i$ and
  $n_k = j$.
  \item Path: a walk in which all nodes/vertices are different
  \item Number of $k$-hop walks between node $i$ and $j$: $(A^k)_{ij}$
  \item Total number of $k$-hop walks in G: $N_k = u^T A^k u = \sum_{i=1}^N \sum_{j=1}^N (A^k)_{ij}$
  \item Total number of closed $k$-hop walks in G:
  $$
  W_k = \sum_{j=1}^N (A^k)_{jj} = \text{trace}(A^k)
  $$
\end{itemize}

\subsubsection{Weight of a path P}
Consider a weighted graph, where each link $l$ has a link weight $w_l$.

The weight of a path P is defined as:
$$
w(P) = \sum_{l \in P} w_l
$$

where $l = n_i \rightarrow n_j$ is a link in $G$ from node $n_i$ to $n_j$ and part of the path
$P = (n_0 \rightarrow n_1)(n_01 \rightarrow n_2)...(n_i \rightarrow n_j)...(n_{h-1} \rightarrow n_h)$
The path $P = P_{n_0 \rightarrow n_h}$ from node $n_0$ to $n_h$ consists of $h$ \emph{different, 
consecutive} links

\subsubsection{Hopcount}
\begin{itemize}
  \item Hopcount of a path $P$ is $h(P) = \sum_{l \in P} 1$, this link weight $w_l = 1$. 
  \item Hopcount from node $i$ to node $j$: 
  $H_{i \rightarrow j} = \min_{P_{i \rightarrow j}} h(P_{i \Rightarrow j}) 
  = h(P_{i \rightarrow j}^{*})$ where $P_{i \rightarrow j}^*$ is the shortest hop path from
  $i$ to $j$
  \item Number of $k$-hop walks between node $i$ and $j$: $(A^k)_{ij}$
\end{itemize}

Algebraic algorithm to find the shortest path
$$
\begin{pmatrix}
  (A)_{ij} = 0 \\
  (A^2)_{ij} = 0 \\
  \vdots \\
  (A^{k -1})_{ij} = 0 \\
  (A^k)_{ij} = m > 0
\end{pmatrix}
$$

The sequence tells us that there are no walks between $i$ and $j$ with less than $k$ hops.

Hence: the shortest walk between $i$ and $j$ is also a shortest path, with hopcount $H_{ij} = k$

There are $m$ different shortest paths with $k$ hops

\subsubsection{Hopcount \& Distance matrix}
A distance matrix $H$ satisfies vector norm or distance relations:
\begin{enumerate}
  \item $h_{ij} \ge 0$
  \item zero diagonal elements: $h_{ij} = 0$
  \item triangle inequality: $h_{ik} + h_{kj} \ge h_{ij}$
\end{enumerate}

\textbf{Diameter} $\rho$: 
\begin{itemize}
  \item Sequence with most zeros (or maximum $k$)
  \item All elements in $\sum_{k =0}^{\rho} f_k A^k$ with all $f_k > 0$ are positive
\end{itemize}

Choose $f_k = \begin{pmatrix}
  \rho \\ k
\end{pmatrix} > 0$
then $\sum_{k = 0}^{\rho} \begin{pmatrix}
  \rho \\ k
\end{pmatrix} A^k = (I + A)^{\rho}$.

The \textbf{small world problem}:
\begin{itemize}
  \item 160 letters: 1 target (name, address, occupation, personal info)
  \item 160 random starters
  \item Each starter forwards a letter to a single acquaintance
  \item Outcome experiment: Average hops of path = 6: everybody is connected by (on average) 6 hops.
\end{itemize}

\subsubsection{Betweenness}

The betweenness $B_l(B_n)$ of a link $l$ (node $n$) equals the number of shortest paths traversink
link $l$ (node $n$) in $G$).

The average betweenness is related to the average hopcount:
$$
H_G = \sum_{i = 1}^N \sum_{j = i + 1}^N H_{ij} = \sum_{l = 1}^L B_l
\implies
E[B_l] = \frac{1}{L} \sum_{l = 1}^L B_l = \frac{1}{L}
\begin{pmatrix}
  N \\ 2
\end{pmatrix}
E[H_G] \ge E[H_G]
$$

If there are many shortest paths we need to adjust the definition. If there are $k$ shortest
paths you count each one of them as $\frac{1}{k}$ 

\subsubsection{Linear correlation coefficient}
Consider $n$ realizations $\{(x_i, y_i)\}_{1 \le i\le m}$ of two random variables $X$ and $Y$.

Goodness of the fit can be expressed by the linear correlation coefficient
$$
\rho(X, Y) = \frac{E[XY] - E[X]E[Y]}{\sqrt{Var[X]}\sqrt{Var[Y]}}
$$

\begin{itemize}
  \item $\rho(X,Y) = 0$: No linear correlation
  \item $\rho(X, Y) = 1$: Perfect fit
\end{itemize}

\subsubsection{Degree assortativity}

Correlation between the left part of the link and the right part. How are the degrees $D_i$ 
and $D_j$ at both sides of a link $l$ correlated?

\begin{itemize}
  \item A network is assortative if $\rho_D > 0$
  \item A network is disassortative if $\rho_D < 0$
\end{itemize}

$$
\rho_D = \frac{N_1 N_3 - N_2^2}{N_1 \sum_{j=1}^N d_j^3 - N_2^2}
$$

Where $N_k = u^T A^k u$ is the total number of walks with $k$ hops

\subsubsection{Degree preserving rewiring}
Between two pairs of nodes you keep the degree, however you rewire the network, and, by doing
that you change the shape of the network and the connectivity. How many rewiring do you have to 
make in order to change the network from assortative to disassortative?

\subsubsection{Connectivity of the complement $G^C$}
If the graph G is disconnected, then ist complement $G^C$ is connected

