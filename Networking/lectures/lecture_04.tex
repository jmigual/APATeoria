% !TEX root = ../main.tex

\section{Network models}

\subsection{Deterministic networks}

\begin{itemize}
  \item Complete graph $K_N$ on $N$
  \item Lattices (in $D$-dimensions): each node has 2D neighbours
\end{itemize}

\subsection{Measurements of networks}

\begin{itemize}
  \item Choose 8 persons
  \item Link = friendship
  \item Properties of friendship graphs?
\end{itemize}

Why do we need this? Reality is just a measurement. You have a lot of friendship networks,
it depends on the realization. 

Motivation to study \emph{random graphs}
e.g. realizations of $G_{0.4}(8)$

\subsection{Complex network models}

\begin{itemize}
  \item Random Graph - Erdös-Rényi (1959-1960)
  \item Small-World Graph - Watts-Strogatz (1998). Introduces the idea of rewiring
  \item Scale-Free Graph - Barabási-Albert (1999)
\end{itemize}

\subsection{Erdös-Rényi random graph}
The ER random graph $G_p(N)$ is a graph with $N$ nodes and each node pair is connected 
independently with probability $p$

Any so generated graph belongs to the class ER random graph $G_p(N)$ with same $N$ and $p$.

There are two variants of ER random graph:
\begin{itemize}
  \item $G_p(N)$ link existence probability $p$
  \item $G(N,L)$ precisely $L$ links?
\end{itemize}

How would you describe or deduce the adjacency matrix $A$ for $G_p(N)$?
$a_{ij}$ (with $j \ne i$) is a Bernoulli random variable with mean $p$

\begin{align*}
  Pr[a_{ij} = 1] &= p \\
  Pr[a_{ij} = 0] &= 1 - p \\
  E[a_{ij}] &= p
\end{align*}
A is a $N \times N$ matrix

The complement graph of $G_p(N)$ is $G_{1-p}(N)$

\begin{itemize}
  \item If $p=1$, then $G_1(N)$ is the complete graph $K_N$
  \item If $p=0$, then $G_0(N)$ is the empty or null graph
  \item The average number of links is
  \begin{align*}
    E[L] = \frac{N(N-1)}{2}p \implies p = \frac{L}{L_{\max}} = \text{link density}
  \end{align*}
  \item The average clustering coefficient is:
  \begin{align*}
    E[c_{G_p(N)}] = p
  \end{align*}
  \item Degree distribution: Binomial distribution
  \begin{align*}
    Pr[D_{rg} = k] &= 
    \begin{pmatrix}
      N - 1 \\ k
    \end{pmatrix}
    p^k ( 1- p)^{N-1-k} \simeq 
    \frac{1}{\sigma \sqrt{2\pi}}e^{-\left( \frac{(k-\mu)^2}{2\sigma^2} \right)} \\
    \mu &= E[D] = (N-1)p \\
    \sigma^2 &= Var[D] = (N-1)p(1 - p)
  \end{align*}
\end{itemize}

\subsubsection{Distribution of eigenvalues of adjacency matrix of $G_p(N)$}

Consider set of eigenvalues as realizations of eigenvalue random variable $\lambda$

Histogram approximates the probability density function $f_{\lambda}(x)$ of eigenvalues
of the adjacency matrix $A$. The largest eigenvalue of a ER graph is really large compared to 
the other ones, the distance is called spectral value.

\subsubsection{Observations from the spectrum}

\begin{itemize}
  \item The probability density function $f_{\lambda}(x)$ of the eigenvalues of the adjacency
  matrix
  \begin{itemize}
    \item Peaks refer to a specific structure or pattern in the graph
    \item A broader, bell-shape form of $f_{\lambda}(x)$ around the origin ($x=0$) is
    a fingerprint of randomness
    \begin{itemize}
      \item Broadening of peaks in spectra to bell-shape forms i mostly due to randomness
    \end{itemize}
  \end{itemize}
  \item If $f_{\lambda}(-x) = f_{\lambda}(x)$ an even function of $x$, then the graph is a tree
  (no triangles, thus skewness $s_{\lambda} = 0$)
  \begin{itemize}
    \item Any tree can be represented by a bipartite graph
    \item A bipartite graph has a symmetric spectrum (for each eigenvalue $\lambda$, there
    exists an eigenvalue $-\lambda$)
  \end{itemize}
\end{itemize}

\subsection{Power law graphs}

Usually, we prefer normalized random variables with mean 0 and variance 1.

A power law degree distribution is also called scale-free:

Any number $a$ just multiplies the probability density; there is no characteristic length. 
The mean is not representative, because the variance is large.

The simplest family of ``power law'' graphs have been proposed by Barbasi-Albert:

\begin{itemize}
  \item Start with $n << N$ nodes
  \item Attach a new node with $m$ links; each link to an already existing node 
  randomly and proportionally to its degree
  \item Repeat 2) until size $N$ is reached
\end{itemize}

\subsubsection{Power law exponent $\tau$ in $Pr[D=k] \simeq ck^{-\tau}$}

\begin{itemize}
  \item Critical point 1
  \begin{itemize}
    \item $E[D]$ diverges
    \item $E[D^2]$ diverges
    \item $D_{max}$ grows faster than $N$
    \item $E[H] \sim $ const
  \end{itemize}
  \item Critical point 2 (needs to be fixed)
  \begin{itemize}
    \item $E[D]$ finite
    \item $E[D^2]$ diverges
    \item $D_{max}$ grows faster than $N$
    \item $E[H] \sim log(log N)$
  \end{itemize}
  \item Critical point 3 (needs to be fixed)
  \begin{itemize}
    \item $E[D]$ diverges
    \item $E[D^2]$ diverges
    \item $D_{max}$ grows faster than $N$
    \item $E[H] \sim $ const
  \end{itemize}
\end{itemize}

\subsection{Observed common properties of real-world complex networks}

\begin{itemize}
  \item small-world property
  \begin{itemize}
    \item Average length/hopcount of a path is short compared to the size $N$ of the network
    ($E[H] = O(log N))$
  \end{itemize}
  \item scale-free degree distribution
  \begin{itemize}
    \item heavy tails (non-Gaussian behavior)
  \end{itemize}
  \item clustering and community structure
  \begin{itemize}
    \item network of networks
  \end{itemize}
  \item robustness to random node failure
  \item vulnerability to targeted hub attacks and cascading failures
\end{itemize}









