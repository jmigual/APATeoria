\documentclass[a4paper]{article}

\usepackage{amsmath}
\usepackage{amssymb}
\usepackage{amsfonts}
\usepackage[catalan]{babel} % Language 
\usepackage{fontspec}
\usepackage[margin=2cm]{geometry}
\usepackage{graphicx}
\usepackage[makeroom]{cancel}
\usepackage{float}

\setlength{\parindent}{0pt}
\setlength{\parskip}{0.2cm}

\title{Tema 4: Models lineals per regressió}
\author{Joan Marcè Iguals}

\begin{document}
\maketitle

\section{Introducció}
El problema de la regressió consisteix en, donat un vector $x \in \mathbb{R}^d$, predir un escalar $t \in \mathbb{R}$, on la relació (la dependència) entre $t$ i $x$ és estocàstica, així doncs escrivim:


\begin{align*}
t = f(x) + \varepsilon,\ \varepsilon &\text{ variable aleatòria} \\
i) \quad &\mathbb{E}(\varepsilon) = 0 \\
ii)\quad & Var(\varepsilon) = \sigma^2 < \infty
\end{align*}

Veiem unes dades:

$$
\mathcal{D} = \{ (x_1, t_1), ... (x_n, t_n)  \} \quad x_n \in R^d, t_n \in R^d \quad m.a.s.
$$

\textbf{Exemple:} ajust polinòmic

$$
y(x, c) = co + \sum_{i=1}^M c_i x^i \quad x \in \mathbb{R}
$$
$$
x \in \mathbb{R}^d\ ? \text{ els polinomis \textbf{no escalen} a més d'una variable}
$$

Introduïm el concepte de \textbf{funcions de base}

$$
\phi_i: \mathbb{R}^d \rightarrow \mathbb{R}, \quad i = 1,...,M
$$
\begin{align*}
y(\vec{x}, \vec{\omega}) &= \omega_0 + \sum_{i=1}^{M-1} \omega_i \phi_i (\vec{x}) \\
&=  \sum_{i=0}^{M-1} \omega_i \phi_i (\vec{x}), \quad \phi_0 (\vec{x}) = 1
\end{align*}
\textbf{Exemple:} $\phi_i (x) = x^i$

Escrivim
$$
y(\vec{x}, \vec{\omega}) = \sum_{i=0}^{M-1} \omega_i \phi_i (\vec{x}) = \vec{\omega}^T \vec{\phi} (\vec{x})
$$

$$
\vec{\omega} = 
\begin{pmatrix}
\omega_0 \\ \omega_1 \\ \vdots \\ \omega_{M-1}
\end{pmatrix};\quad
\vec{\phi}(\vec{x}) = 
\begin{pmatrix}
\phi_0 (\vec{x}) = 1 \\
\phi_1 (\vec{x}) \\
\vdots \\
\phi_{M-1} (\vec{x})
\end{pmatrix}
$$

\section{Regressió clàssica}

$$
\varepsilon \sim N(0, \sigma^2)
$$

Si tenim una distribució normal (veure figura 1) l'amplada del pic de la gaussiana depèn de $\sigma^2$.


% FIGURA 1

$ t = f(x) + N(0, \sigma^2) $ és equivalent a escriure $T \sim N(f(x), \sigma^2) $.

Així doncs si s'agafa la mitjana de $x$ i s'uneixen els punts, la funció resultant és la millor possible. Així doncs la funció de regressió $f^*(x) = \mathbb{E}(T|X)$.

L'únic que ens falta és posar les dades en una matriu:

$$
\vec{X}_{N \times d} =
\begin{pmatrix}
\longleftarrow x_1 \longrightarrow \\
\longleftarrow x_2 \longrightarrow \\
\vdots \\
\longleftarrow x_N \longrightarrow
\end{pmatrix}
$$

D'aquesta manera tenim $\mathcal{D}$ conegudes $(x_n, t_n)$, $\omega$ i $\sigma^2$ desconeguts, per tant apliquem \textbf{Màxima Versemblança}.

$$
\theta = (\omega, \sigma^2)
$$

\begin{gather*}
-\ell (\theta) = -ln \mathcal{L} (\theta) = 
-ln \mathcal{L}(\omega, \sigma^2) =
-ln P(t | X, \omega, \sigma^2) \underbrace{=}_{m.a.s.} \\
-ln \prod_{n=1}^N P(t_n | x_n, \omega, \sigma^2) \underbrace{=}_{modelat}
-ln \prod_{n=1}^N \mathcal{N} (t_n, y(x_n, \omega), \sigma^2) =
- \sum_{n=1}^N ln \ \mathcal{N} (t_n, y(x_n, \omega), \sigma^2) = \\
- \sum_{n=1}^N \left\{ ln \left( \frac{1}{\sqrt{2\pi}\sigma} \right) - \frac{(t_n - y(x_n, \omega))^2}{2\sigma^2} \right\} = 
\sum_{n=1}^N \left\{ ln(\sqrt{2\pi}\sigma) + \frac{(t_n - y(x_n, \omega))^2}{2\sigma^2} \right\} = \\
N ln (\sqrt{2\pi}) + \frac{1}{2\sigma^2} \sum_{n=1}^N (t_n - y(x_n,\omega))^2 \text{ és equivalent a } \frac{1}{2} \sum_{n=1}^N (t_n - y(x_n, \omega))^2 = E(\omega)
\end{gather*}

$$
\boxed{y(x, \vec{\omega}) = \vec{\omega}^T \Phi(x)} \text{ és convinent }
$$

$$
\Phi_{N \times M} =
\begin{pmatrix}
\phi_0 (x_1) & \phi_1(x_1) & \ldots & \phi_{M-1} (x_1) \\
\phi_0 (x_2) & \phi_1(x_2) & \ldots & \phi_{M-1} (x_2) \\
\vdots & \vdots & \ddots & \vdots \\
\phi_0 (x_N) & \phi_1(x_N) & \ldots & \phi_{M-1} (x_N)
\end{pmatrix}
\text{ matriu de disseny}
$$

$$
\mathcal{N}(t, \mu, \sigma^2) = \frac{1}{\sqrt{2 \pi} \sigma} exp \left( - \frac{(t - \mu)^2}{2\sigma^2} \right)
$$

Expressem \textbf{l'error quadràtic}
$$
E(\omega) = \frac{1}{2} || \vec{t} - \Phi \vec{\omega} ||^2 = \frac{1}{2} \left\{ ||\vec{t}||^2 + ||\Phi \vec{\omega}||^2 - 2t^T(\Phi \vec{\omega}) \right\}
$$

$$
\frac{\partial E}{\partial \vec{\omega}} = \frac{1}{2} \left\{ 0 + 2 \Phi^T \Phi \hat{\vec{\omega}} - 2 \Phi^T \vec{t} \right\} = \Phi^T \Phi \hat{\vec{\omega}} - \Phi^T \vec{t} = \vec{0} \implies
$$
$$
\left( \Phi^T \Phi \right) \hat{\vec{\omega}} = \Phi^T \vec{t} \implies
\boxed{\hat{\vec{\omega}} = (\Phi^T \Phi)^{-1} \Phi \vec{t}} = \Phi^+ \vec{t}
$$
$$
\frac{\partial E}{\partial \sigma^2} = \frac{1}{N} \sum_{n=1}^N (t_n - \vec{\omega}^T \phi(x_n))^2 = \frac{2}{N} E(\omega)
$$

\textbf{Teorema} ("mínims quadrats")
Sigui $\Phi_{N\times M}, \quad N > M$. Si els vectors columna de $\Phi$ són linealment independents, és a dir, $rang(\Phi) = M$ (\emph{full rank}), llavors:

\begin{enumerate}
	\item La matriu $\Phi^T\Phi$ és simètrica i semi-definida positiva
	\item El problema de mínims quadrats: 
	$\underset{\vec{\omega}}{min} ||\vec{t} - \Phi \vec{\omega}||^2$ té solució única.
	\item Per trobar-la, cal resoldre el sistema d'equacions: $\Phi^T\Phi \vec{\omega} = \Phi^T \vec{t}$ (equacions normals de'n Gauss)
\end{enumerate}

La matriu $ \Phi^+ := \left( \Phi^T \Phi \right)^{-1} \Phi^T $ és pseudo-inversa (inversa de Moore-Penrose) $ \Phi^+ \Phi = I $

\end{document}