\documentclass[a4paper]{article}

\usepackage{amsmath}
\usepackage{amssymb}
\usepackage{amsfonts}
\usepackage[catalan]{babel} % Language 
\usepackage{fontspec}
\usepackage[margin=2cm]{geometry}
\usepackage{graphicx}
\usepackage[makeroom]{cancel}
\usepackage{float}

\setlength{\parindent}{0pt}
\setlength{\parskip}{0.2cm}

\title{Tema 4: Models lineals per regressió}

\begin{document}
\maketitle

El problema de la regressió consisteix en, donat un vector $x \in \mathbb{R}^d$, predir un escalar $t \in \mathbb{R}$, on la relació (la dependència) entre $t$ i $x$ és estocàstica, així doncs escrivim:


\begin{align*}
t = f(x) + \varepsilon,\ \varepsilon &\text{ variable aleatòria} \\
i) \quad &\mathbb{E}(\varepsilon) = 0 \\
ii)\quad & Var(\varepsilon) = \sigma^2 < \infty
\end{align*}



\end{document}