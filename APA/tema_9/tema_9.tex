\chapter{Mètodes de conjunt (``ensemble")}
\section{Introducció}

\begin{itemize}
	\item Un conjunt de models pot arribar a generalitzar millor que un dels models del conjunt triat a l'atzar.
	\item Quan un model és dolent:
	\begin{itemize}
		\item té massa biaix (infrajustat)
		\item té massa variància (sobreajustat)
	\end{itemize}
\end{itemize}

Hi ha dos grans grups de mètodes per resoldre aquests problemes:
\begin{itemize}
	\item \textbf{Bagging}: Per disminuir la variància
	\item \textbf{Boosting}: Per reduir el biaix
\end{itemize}

\section{Bagging (Bootstrap Aggregating)}
Si es té tot un conjunt de models i es volen integrar en un de sol, els models individuals han de presentar poc biaix i molta variància. 

\begin{itemize}
	\item Arbres de decisió
	\item Xarxes neuronals (cost computacional gran)
\end{itemize}

\begin{itemize}
	\item Es vol generar diversitat entre els models individuals
	\item Idealment, es necessitaria una mostra de dades \textbf{diferent} per cada model individual.
	\item No es disposa d'aquesta quantitat de mostres per tant es re-mostreja la única mostra que es té.
\end{itemize}


Imaginem que es té un conjunt de dades $\mathcal{D}$

\begin{align*}
	&\mathcal{D} = \{ (x_1, t_1),..., (x_N, t_N) \} \\
	& \hat{\theta} = \theta(\mathcal{D}) \text{ del veritable (poblacional) }, \theta \\
	& Z_1^* \{ 1, ..., N \} \ \text{(amb re-emplaç)}\ \hat{\theta}_1^* = \theta(Z_1^*) \\
	& Z_2^* \{ 1, ..., N \} \ \text{(amb re-emplaç)}\ \hat{\theta}_2^* = \theta(Z_2^*) \\
	& \vdots \\
	& Z_B^* \{ 1, ..., N \} \ \text{(amb re-emplaç)}\ \hat{\theta}_B^* = \theta(Z_B^*) \\
\end{align*}
$$
\hat{\theta}^* = \frac{1}{B} \sum_{i=1}^B \hat{\theta}_i^*
$$
