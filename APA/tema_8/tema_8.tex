\documentclass[a4paper]{article}

\usepackage{amsmath}
\usepackage{amssymb}
\usepackage{amsfonts}
\usepackage{mathtools}
\usepackage[catalan]{babel} % Language 
\usepackage{fontspec}
\usepackage[makeroom]{cancel}

\setlength{\parindent}{0pt}
\setlength{\parskip}{0.2cm}

\title{Tema 8: Màquines de vectors suport (Introducció als mètodes kernel)}
\author{Joan Marcè i Igual}

\begin{document}
\section{Introducció}

Estem treballant amb models de la forma:
$$
y(x) = g(w^T\phi(x) + w_0),\ x \in \mathbb{R}^d, w \in \mathbb{R}^M, w_0 \in \mathbb{R}^d
$$
$$
\phi(x) =
\begin{pmatrix}
\phi_1(x) \\
\phi_2(x) \\
\vdots \\
\phi_M(x)
\end{pmatrix}
$$

En xarxes neuronals es va dir que $\phi_i(x) = \phi(\varphi(x, v_i))$. 

Ara calculem un model lineal en l'espai de representació generat per les $\phi_i(x)$ (\textbf{feature space} = espai de característiques). El que fan els mètodes no lineals és buscar l'espai d'estadístiques adequat per a un problema en concret.

Una manera diferent de treballar és estudiar l'expressió:
\begin{align*}
	\phi(x)^T\phi(y) \quad x,y \in \mathbb{R}^d \\
	\phi: \text{feature map} 
\end{align*}

\textbf{Definim:} una funció de kernel $k: \mathbb{R}^d \times \mathbb{R}^d \rightarrow \mathbb{R}$ com:
$$
k(x,y) := \phi(x)^T\phi(y)
$$

\begin{itemize}
	\item Es poden estudiar les propietats de $k$ sense calcular (ni tan sols conèixer) $\phi$ de manera explícita.
	\item Es pot obtenir un algorisme d'aprenentatge \textbf{no lineal} partir d'un de lineal mitjançant l'ús de funcions de kernel (\emph{kernelitzar}).
	$$
	x^Ty \rightarrow \underbrace{\phi(x)^T\phi(y)}_{\mathclap{\text{$\phi$ no lineal, el resultat és un algorisme no lineal}}} = k(x,y)
	$$
\end{itemize}

\section{Com construir funcions de kernel}
Si tenim un 
$$
\phi(x) = 
\begin{pmatrix}
\phi_1(x)\\
\phi_2(x) \\
\vdots \\
\phi_n(x)
\end{pmatrix}
$$
$$
\implies k(x,y) = \phi^T(x)\phi(x) = \sum_{i=1}^M \phi_i(x)\phi_i(y)
$$

\textbf{Exemple} $(x^Ty)^3 \ ,x,y \in \mathbb{R}^3$
$$
x = \begin{pmatrix}
x_1 \\
x_2 \\
x_3
\end{pmatrix};
\quad 
y = \begin{pmatrix}
y_1 \\
y_2 \\
y_3
\end{pmatrix}
$$
\begin{align*}
	& (x^Ty)^3 = (x_1y_y + x_2y_2 + x_3y_3) = \\
	& (x_1y_1)^3 + 3(x_1y_1 + x_2y_2)^2 x_3y_3 + 3(x_1y_1 +
	 x_2y_2)(x_3y_3)^2 + (x_3y_3)^3 = ... \\
	& = 
	\begin{pmatrix}
	x_1^3 \\
	x_2^3 \\
	x_3^3 \\
	\sqrt{6} x_1x_2x_3 \\
	\sqrt{3} x_1^2x_2 \\
	\sqrt{3} x_1^2x_3 \\
	\sqrt{3} x_2^2x_1 \\
	\sqrt{3} x_2^2x_3 \\
	\sqrt{3} x_3^3x_1 \\
	\sqrt{3} x_3^3x_2 \\
	\end{pmatrix}^T 
	\begin{pmatrix}
	y_1^3 \\
	y_2^3 \\
	y_3^3 \\
	\sqrt{6} y_1y_2y_3 \\
	\sqrt{3} y_1^2y_2 \\
	\sqrt{3} y_1^2y_3 \\
	\sqrt{3} y_2^2y_1 \\
	\sqrt{3} y_2^2y_3 \\
	\sqrt{3} y_3^2y_1 \\
	\sqrt{3} y_3^2y_2
	\end{pmatrix}
	= \phi(x)^T\phi(y) \qquad \phi:\mathbb{R}^3 \rightarrow \mathbb{R}^{10}
\end{align*}

% FIGURA 2

$$
(x^Ty + 1)^3 \implies
$$
$\phi$ calcula tots els monomis de les variables (1,...,d) fins a grau 3. (0, 1, 2, 3).

$x^Ty$ \textbf{kernel lineal}

$\phi(x) = x$

$(x^Ty + 1)^2$ kernel quadràtic

$(x^Ty + 1)^3$ kernel cúbic

$(x^Ty + c)^q \quad ,c \ge 0, q \in \mathbb{N}$ kernel polinòmic

\textbf{Teorema} Una funció $k:\mathbb{R}^d \times \mathbb{R}^d \rightarrow \mathbb{R}$ és una \textbf{funció lineal} ($\exists \phi: \mathbb{R}^d \rightarrow \mathbb{R}^D, t.q. \ k(x,y) = \phi^T(x)\phi(y)$) si i només si:
\begin{itemize}
	\item $k$ és simètrica
	$$
		k(x,y) = k(y,x) \quad ,\forall x,y \in \mathbb{R}^d
	$$
	\item $\forall N \in \mathbb{N}$, per tota tria $x_1,...,x_N \ \mathbb{R}^d$, la matriu $K_{i,j} := k(x_i, x_j)$ és semi-definida positiva (PSD)
\end{itemize}

\textbf{Definició:} diem que una matriu $K_{N \times N}$ és PSD si és simètrica i $\forall c \in \mathbb{R}^N, c^T K c \ge 0$. És a dir $\sum_{i=1}^N\sum_{j=1}^N c_i K_{ij} c_j \ge 0$.

\textbf{Demostració:} Si la funció $k$ és un kernel $\implies K$ és PSD.
\begin{align*}
	& \sum_i \sum_j c_i K_{ij} c_j = 
	\sum_i \sum_j c_i \phi^T(x_i) \phi(x_j)c_j = \\
	& \left( \sum_i c_i \phi(x_i) \right)^T
	\left( \sum_j c_j \phi(x_j) \right) = 
	\left|\left| \sum_i c_i \phi(x_i) \right|\right|^2 \ge 0
\end{align*}

\textbf{Propietats}
Si $k$, kernels, $a \in \mathbb{R} \quad a \ge 0$:
\begin{itemize}
	\item $a·k$ kernel
	\item $k_1 + k_2$ kernel
	\item $k_1·k_2$ kernel
	\item p(k) kernel, si p és un polinomi de coeficients no negatius
\end{itemize}

\textbf{Proposició}
$$
k(x,y) = (x^Ty + c)^q \quad \forall c \ge 0, \forall q \in \mathbb{N}
$$

Podem definir funcions de kernel en qualsevol conjunt d'objectes o espai.
$$
k: \mathbb{R}^d \rightarrow \mathbb{R}^d 
\rightarrow \mathbb{R} \implies 
k: X \times X \rightarrow \mathbb{R}
$$
$X$ conjunt d'objectes finit (persones).

Una manera de comparar dos conjunts és comparar quants elements tenen en comú
$$
k(x,y) = | x \cap y|
$$
$$
x,y \subset X
$$
$$
I_X (M) = \begin{cases}
1 & \text{ si } M \in X \\
0 & \text{ si } M \cancel{\in} X
\end{cases}
$$
$$
|x \cap y| = \sum_{u \in X} I_X(u)·I_y(u) = k(x,y)
$$

\end{document}
