\documentclass[a4paper]{report}
\usepackage[margin=3cm]{geometry}

\usepackage{amsmath}
\usepackage{amssymb}
\usepackage{amsfonts}
\usepackage{mathtools}
\usepackage[catalan]{babel} % Language 
\usepackage{fontspec}
\usepackage[makeroom]{cancel}
\usepackage[dvipsnames]{xcolor}
\usepackage{tikz}
\usepackage{pgfplots}
\usepackage{float}
\usepackage{graphicx}
\usepackage[hidelinks]{hyperref}
\usepackage{algpseudocode}
\usepackage{algorithmicx}
\usepackage{enumerate}
\usepackage{listings}
\usepackage{ifthen}
\usepackage{array}

\pgfplotsset{compat=1.13}

\usetikzlibrary{patterns}
\usetikzlibrary{positioning}

\tikzset{%
	every neuron/.style={
		circle,
		draw,
		minimum size=1cm
	},
	neuron missing/.style={
		draw=none, 
		scale=4,
		text height=0.333cm,
		execute at begin node=\color{black}$\vdots$
	},
}

\setcounter{secnumdepth}{5}

% New commands definitions
\newcommand{\verteq}{\rotatebox{90}{$\,=$}}
\newcommand*{\op}[1]{\operatorname{#1}}
\newcommand*{\bmath}[1]{\boldsymbol{#1}}


\title{Aprenentatge Automàtic}

\begin{document}

\maketitle

\setlength{\parskip}{0.6em}
\tableofcontents
\pagebreak

\setlength{\parindent}{0pt}
\setlength{\parskip}{1em}

\input{tema_1/tema_1.tex}
\input{tema_2/tema_2.tex}
\documentclass[a4paper]{article}

\usepackage{amsmath}
\usepackage{amssymb,amsfonts}
\usepackage[catalan]{babel} % Language 
\usepackage{fontspec}
\usepackage[margin=2cm]{geometry}
\usepackage{graphicx}

\title{Tema 3: Mètodes de \emph{clustering}}

\begin{document}
	
\maketitle

\textbf{Objectiu}: El nostre objectiu és trobar agrupacions naturals de dades. Un grup o subgrup de dades és un \emph{cluster}

Les observacions que pertanyen al mateix \emph{cluster} s'assemblen entre elles

Les observacions que \textbf{no} pertanyen al mateix \emph{cluster} no s'assemblen tant

\textbf{\emph{clustering}}

\begin{itemize}
	\item el \textbf{procés} de trobar els \emph{clusters} presents en les dades
	\item el \textbf{resultat} del procés.
\end{itemize}

% Figures 1 i 2

\subsection{Mètodes de \emph{clustering}}

Un tipus de mètode són els mètodes jeràrquics:
\begin{itemize}
	\item \textbf{divisius} es dediquen a separar les dades en dos grups recursivament fins que només queda un punt.
	\item \textbf{aglomeratius} es dediquen a agrupar les dades des d'un punt fins a obtenir el \emph{cluster} final.
\end{itemize}

% Figura 3

També hi ha els mètodes combinatoris. Hi ha un algoritme voraç amb una funció de qualitat.

Hi ha els algoritmes probabilístics. De quantes maneres es poden agrupar $N$ dades en $K$ \emph{clusters}?

$$ S(N, K) = \frac{1}{K!} \sum_{k=1}^K (-1)^{K-k} \begin{pmatrix}
K \\ k
\end{pmatrix} \text{, nº d'Stirling del 2n tipus} $$

$$ S(19,4) \simeq = 10^10 $$

$$ \underbrace{B(N)}_{\text{nº de Bell}} \sum_{K=1}^{N} S(N,K) \implies \text{ ex.: } B(79) \simeq 3.89·10^85 $$

Així doncs hi ha diferents mètodes de \emph{clustering} per trobar els grups.

\subsection{Algorisme de $k$-means}

Tenim una mostra $\mathcal{D} = \{ x_1,..., x_N \} , x_i \in \mathbb{R}, 1 \le i \le N$. Fixa't $K$ externament, escollim un conjunt de $K$ \textbf{prototips}:

$$ \mathcal{P} = \{ \mu_1, ..., \mu_K \}, \mu \in \mathbb{R}^d $$

\end{document}
\documentclass[a4paper]{article}

\usepackage{amsmath}
\usepackage{amssymb}
\usepackage{amsfonts}
\usepackage[catalan]{babel} % Language 
\usepackage{fontspec}
\usepackage[margin=2cm]{geometry}
\usepackage{graphicx}
\usepackage[makeroom]{cancel}
\usepackage{float}

\setlength{\parindent}{0pt}
\setlength{\parskip}{0.2cm}

\title{Tema 4: Models lineals per regressió}

\begin{document}
\maketitle

El problema de la regressió consisteix en, donat un vector $x \in \mathbb{R}^d$, predir un escalar $t \in \mathbb{R}$, on la relació (la dependència) entre $t$ i $x$ és estocàstica, així doncs escrivim:


\begin{align*}
t = f(x) + \varepsilon,\ \varepsilon &\text{ variable aleatòria} \\
i) \quad &\mathbb{E}(\varepsilon) = 0 \\
ii)\quad & Var(\varepsilon) = \sigma^2 < \infty
\end{align*}



\end{document}
\documentclass[a4paper]{article}
%\usepackage[margin=2cm]{geometry}

\usepackage{amsmath}
\usepackage{amssymb}
\usepackage{amsfonts}
\usepackage{mathtools}
\usepackage[catalan]{babel} % Language 
\usepackage{fontspec}
\usepackage{graphicx}
\usepackage[makeroom]{cancel}
\usepackage{float}
\usepackage{enumerate}
\usepackage{pgfplots}
\usepackage[hidelinks]{hyperref}

\pgfplotsset{compat=1.13}

\setlength{\parindent}{0pt}
\setlength{\parskip}{0.2cm}

\title{Tema 5: Classificadors generatius}
\author{Joan Marcè Igual}

\begin{document}
	
\maketitle

\section{Introducció}

Un \textbf{classificador} és una funció:
$$
y:\mathbb{R}^d \rightarrow \{ 1,2,...,K \}
$$
$$
K = 2 \rightarrow \text{classificació binària}
$$

Ens interessarà molt trobar mètodes que ofereixin probabilitats. Per exemple:
\begin{align*}
	K = 3 \qquad &x \rightarrow y(x) = "2" \quad (0\ 1\ 0) \quad \text{\textbf{HARD}}\\
	& x \rightarrow y(x) = (0.2, 0.5, 0.3) \quad \text{\textbf{SOFT}}
\end{align*}

Un classificador parteix $\mathbb{R}^d$ en $K$ subconjunts. 

\textbf{Regions de decisió:}

\begin{figure}[H]
	\centering
	\includegraphics[width=0.5\textwidth]{images/tema_5-1}
\end{figure}

Les regions de decisió estan generades per les zones que delimiten cada classe. Hi ha unes fronteres on les regions canvien. Les separacions es diuen \emph{fronteres de decisió}.

Un classificador és \textbf{lineal} quan només genera fronteres de decisió (entre cada parell de classes) lineals. Per exemple:

\begin{figure}[H]
	\centering
	\includegraphics[width=0.5\textwidth]{images/tema_5-2}
\end{figure}

\subsection{La formula de'n Bayes}

Suposem que tenim dues variables aleatòries $A$, $B$, que prenen valors en:

$$
\{ a_1, ..., a_n \}\quad \{ b_1, ..., b_n \}
$$
$$
\begin{cases}
p(a_k) = \sum_{j=1}^n P(a_k, b_j) = \sum_{j=1}^n P(a_k | b_j) P(b_j) \\
P(a_k, b_j) = P(b_j, a_k)
\end{cases}
$$
$$
P(a_k | b_j) P(b_j) = P(b_j | a_k) P(a_k) \implies
\boxed{P(b_j | a_k) = \frac{P(a_k | b_j) P(b_j)}{\sum_{q=1}^n P(a_k | b_q) P(b_q)}}
$$
$$
\sum_{j=1}^m P(b_j | a_k) = 1,\ \forall k 
$$

\textbf{Exemple:} Tenim dues urnes amb pomes ($P$) i taronges ($T$). La probabilitat d'agafar la urna 1 és $\frac{2}{3}$ i la d'agafar la urna 2 és de $\frac{1}{3}$. S'agafa d'una urna a l'atzar un a taronja, quina és la probabilitat que la taronja provingui de la urna 1?

$$
P(U_1|T) =  \frac{P(T|U_1) P(U_1)}{\underbrace{P(T|U_1)P(U_1) + P(T|U_2)P(U_2)}_{P(T)}} =
\frac{\frac{2}{7} · \frac{2}{3}}{\frac{2}{7}·\frac{2}{3} + \frac{7}{9}·\frac{1}{3}} =
\frac{36}{85} < \frac{1}{2}
$$

\begin{figure}[H]
	\centering
	\includegraphics[width=0.7\textwidth]{images/tema_5-3}
\end{figure}

\textbf{Exemple introductori:} Imaginem que tenim un amic ornitòleg que pren mesures de dos tipus d'aus \emph{àligues} i \emph{falcons} i pren mesures de la envergadura (en cm). El nostre amic vol decidir si a partir de la envergadura una au és una àliga o un falcó. 

\begin{figure}[H]
	\centering
	\includegraphics[width=0.7\textwidth]{images/tema_5-4}
\end{figure}

Es tracta de donar una resposta que sigui la òptima. 
$$
N \text{ dades: }
\begin{rcases*}
N_1 \text{ falcons} \\
N_2 \text{ àligues}
\end{rcases*}
\begin{aligned}
C_1 \text{ (falcons)} \\
C_2 \text{ (àligues)}
\end{aligned}
$$

La mida de cada au no és estàndard sinó que varia en funció de l'entorn de manera aleatòria. 

$$
R_1: \text{ la classe de } x = 
\begin{cases}
C_1 \text{ si } \frac{N_1}{N} > \frac{N_2}{N} \\
C_2 \text{ si } \frac{N_2}{N} < \frac{N_1}{N}
\end{cases}
$$ 
Aquesta regla té força errors perquè és constant i no té en compte l'envergadura que hem estat mesurant.

$$
P_{R_1} (error) = \min\left(\frac{N_1}{N}, \frac{N_2}{N}\right) = \frac{1}{N} \min(N_1, N_2)
$$
$$
P_{R_2} (error) = \min(P(C_1), P(C_2))
$$
\begin{align*}
	& P(C_1 | X) = \frac{P(X|C_1)P(C_1)}{P(X|C_1)P(C_1) + P(X|C_2)P(C_2)} \\
	& P(C_2|X) = 1 - P(C_1|X)
\end{align*}

$$
R_{Bayes}: \text{ la classe de } X =
\begin{cases}
C_1 \text{ si } P(C_1|X) > P(C_2|X) \\
C_2 \text{ si } P(C_1|X) < P(C_2|X) \\
\end{cases}
$$
\begin{align*}
&P_{R_{Bays}} (error|X) = min\{P(C_1|X), P(C_2|X)\} \\
&P(error_{total}) = \mathbb{E}(P_{Bayes} (error|X)) = 
\int_{\mathbb{R}} P_{R_{Bays}} (error|X)P(X)dX =
\end{align*}
$$
\int_{\mathbb{R}} \min\{P(C_1|X),P(C_2|X)\}P(X)dX =
$$
$$
\int_{S} \min\{\frac{P(X|C_1)P(C_1)}{\cancel{P(X)}}, \frac{P(X|C_2)P(C_2)}{\cancel{P(X)}}\}\cancel{P(X)} dX \quad S = \{X|P(X) > 0\} =
$$
$$
\int_S \min\{ P(X|C_1)P(C_1), P(X|C_2)P(C_2) \} dX \le
$$
$$
\min\left\{ \int_S P(X|C_1)P(C_1) dX,\ \int_S P(X|C_2)P(C_2)dX \right\} =
$$
$$
\min\left\{ P(C_1)\int_S P(X|C_1)dX,\ P(C_2)\int_S P(X|C_2)dX \right\} =
$$
$$
\min \{ P(C_1), P(C_2) \} = P_{R_2} (error)
$$

$$
\int_S \min(f_1(x), f_2(x)) dx \le \min\left( \int_S f_1(x) dx,\ \int_S f_2(x) dx \right)
$$

\textbf{Teorema:} La $R_{Bays}$ és la millor regla de decisió (classificació) d'entre les que fan servir $P(C_1|X), P(C_2|X)$.

\section{Classificadors generatius}

$$
P(C_1|X) = \frac{P(X|C_1)P(C_1)}{P(X|C_1)P(C_1) + P(X|C_2)P(C_2)} = 
\frac{1}{1 + \frac{P(X|C_2)P(C_2)}{P(X|C_1)P(C_1)}} =
$$
$$
\text{definim } a(x) := \ln \left( \frac{P(X|C_1)P(C_1)}{P(X|C_2)P(C_2)} \right)
$$
$$
P(C_1|X) = \frac{1}{1 + \exp(-a(x))} = 
$$
$$
\text{definim } g(z) := \frac{1}{1 + e^{-z}} \implies P(C_1|X) = g(a(x))
$$


\subsection{Funció logística}

\begin{figure}[H]
	\centering
	\includegraphics[width=0.8\textwidth]{images/tema_5-5}
\end{figure}

$$
g:\mathbb{R} \rightarrow (0,1)
$$
La funció logística envia tots els valors de $\mathbb{R}$ a un valor comprès entre 0 i 1. Veure \autoref{fig:logistic}

\begin{figure}[H]
	\centering
	\begin{tikzpicture}
	\begin{axis}[
	xmin=-7,xmax=7,ymin=0,ymax=1,
	domain=-10:10
	]
	\addplot+[mark=none] {1/(1 + exp(-x))};
	\end{axis}
	\end{tikzpicture}
	\caption{Gràfic funció logística}
	\label{fig:logistic}
\end{figure}

\begin{itemize}
	\item $ \lim\limits_{z \to +\infty} g(z) = 1 $
	\item $ \lim\limits_{z \to -\infty} g(z) = 0 $
	\item $ g(-z) = 1 - g(z) $
	\item $ g(z) = g(z)[1 - g(z)] $
\end{itemize}

\subsection{Cas de K classes}
$$
1 \le k \le K \qquad P(C_k | X) = \frac{P(X|C_k)P(C_k)}{\sum_{j=1}^K P(X|C_j)P(C_j)} =
$$
$$
\text{definim } a_k(x) = \ln\{P(X|C_k)P(C_k)\}
$$
$$
P(C_k|X) = \frac{\exp(a_k(x))}{\sum_{j=1}^K \exp(a_j(x))} = G_k (a_1(x), a_2(x),...,a_K(x))
$$
$$
G_k(z_1,..,z_K) = \frac{\exp(z_k)}{\sum_{j=1}^N \exp(z_j)} \longrightarrow \text{ softmax}
$$

\section{Aplicació a distribucions Gaussianes}
$$
P(X|C_k) ? \qquad x \in \mathbb{R}^d
$$

Ara suposarem que les dades d'una classe venen d'una Gaussiana multivariable.
$$
P(X|C_k) = N(x, \mu_k, \Sigma_k)
$$

Cas $K=2$
$$
a(x) = \ln \left( \frac{P(X|C_1)P(C_1)}{P(X|C_2)P(C_2)} \right) =
$$
$$
\ln \left( \frac{|\Sigma_1|^{-\frac{1}{2}} 
	\exp\{ -\frac{1}{2}(x - \mu_1)^T \Sigma_1^{-1}(x - \mu_1)^T \} }
	{|\Sigma_2|^{-\frac{1}{2}} 
	\exp\{ -\frac{1}{2}(x - \mu_2)^T \Sigma_2^{-1}(x - \mu_2)^T \}} ·
\frac{P(C_1)}{P(C_2)}\right) =
$$
$$
\frac{1}{2}(x - \mu_2)^T \Sigma_2^{-1} (x - \mu_2) - \frac{1}{2}(x - \mu_1)^T 
\Sigma_1^{-1}(x - \mu_1) + \frac{1}{2}\ln|\Sigma_2| - \frac{1}{2}\ln|\Sigma_1| + 
\ln \frac{P(C_1)}{P(C_2)}
$$

Si $\Sigma_1 = \Sigma_2 = ... = \Sigma_K = \Sigma$ llavors $a_k(x)$ és un \textbf{classificador lineal}.
$$
a_k(x) = w^T x + w_0
$$
\begin{align*}
	& w := \Sigma^{-1}(\mu_1 - \mu_2) \\
	& w_0 := \frac{1}{2} \mu_2^T \Sigma^{-1}\mu_2 - \frac{1}{2}\mu_1^T \Sigma^{-1}\mu_1
	+ \ln \frac{P(C_1)}{P(C_2)}
\end{align*}
$$
P(C_1|X) = g(a(x)) = g(w^Tx + w_0)
$$

\subsection{Estimació dels paràmetres}

\begin{align*}
&	P(C_k) \approx \frac{N_k}{N} \\
&	\mu_k \approx \frac{1}{N_k} \sum_{x_n \in C_k} x_n \\
&	\Sigma_k = \frac{1}{N_k - 1} \sum_{x_n \in C_k} (x_n - \hat{\mu}_k)(x_n - \hat{\mu}_k)^T \\
&	\Sigma_{pooled} = \frac{1}{N - K} \sum_{k=1}^K (N_k - 1) \hat{\Sigma_k}
\end{align*}

\section{Naïve Bayes}
$$
P(C_k|X) = \frac{P(X|C_k)P(C_k)}{P(X)}
$$


$$
P(C_k|X) = \frac{P(x|C_k)P(C_k)}{P(X)} \quad k=1,...,k=K
$$
$$
P(X|C_k) = P(X_1=x_1, X_2=x_2, ..., X_d = x_d|C_k)
$$
$$
x = 
\begin{pmatrix}
x_1 \\ \vdots \\ x_d
\end{pmatrix}, \quad
x \in \mathbb{R}
$$
$$
P(X|C_k) = P(X_1=x_1|C_k)\prod_{j=2}^d P(X_j=x_j|X_1=x_1,X_2=x_2,...,X_{j-1}=x_{j-1}|C_k)
$$
$$
d=3 \rightarrow P(X_1,X_2,X_3) = P(X_3|X_1,X_2)P(X_2|X_1)P(X_1)
$$

$$
P(X|C_k) \underbrace{=}_{\text{naïve}} P(X_1=x_1|C_k) 
\prod_{j=2}^d P(X_j=x_j|C_k) = \prod_{j=1}^d P(X_j=x_j|C_k)
$$
$$
\implies a_k(x) = \ln(P(X|C_k)P(C_k)) = 
\ln\left(P(C_k)\prod_{j=1}^d P(X_j=x_j|C_k)\right) =
$$
$$
= ln P(C_k) + \sum_{j=1}^d \ln P(X_j=x_j|C_k)
$$

\begin{figure}[H]
	\centering
	\includegraphics[width=0.7\textwidth]{images/tema_5-6}
\end{figure}

$X_1$ i $X_2$ semblen dependents a primera vista però si només ens centrem en un dels dos conjunts llavors deixen de ser-ho.



\end{document}
\chapter{Classificadors probabilístics discriminatoris}
\section{Introducció}
Ja s'havia vist anteriorment que \textbf{LDA}:

$$
P(C_k|X) = \frac{P(X|C_k)P(C_k)}{\sum_{j=1}^K P(X|C_j)P(C_j)}\quad k=1,...,K
$$
$$
g(z) = \frac{1}{1 + e^{-z}}
$$

\textbf{Nº de paràmetres:}

$$
\Sigma_{d\times d} =
\begin{pmatrix}
d & & TS \\
& \ddots & \\
TI & & d
\end{pmatrix} \quad TS = TI \qquad 
\boxed{n = \frac{d(d+1)}{2} + Kd \sim O(d^2)}
$$

El nombre de paràmetres creixen quadràticament. De manera que aquests tipus de mètodes només serveixen per conjunts amb moltes dades.

En els classificadors discriminatius no es modelarà la forma de les classes. Es parteix del model $P(C_1|X) = g(w^T X + w_0)$. La idea és optimitzar els $w \in \mathbb{R}^d,w_0 \in \mathbb{R}$ directament (sense assumir cap distribució de les dades condicionades a la classe). Quants paràmetres s'han d'estimar? Doncs $d + 1 \in O(d)$ 

Per simplificar,
$$
w = \begin{pmatrix}
w_0 \\
w_1 \\
\vdots\\
w_d
\end{pmatrix}
\qquad
x = 
\begin{pmatrix}
x_0 = 1\\
x_1\\
\vdots\\
x_d
\end{pmatrix}
$$

Construïm la \textbf{menys log-versemblança de la mostra}:
$$
\mathcal{D} = \{ (x_1,t_1), ..., (x_N,t_N) \} 
\quad x_n \in \mathbb{R}^d
\quad t_n \in \{0, 1\}
\quad
\begin{cases}
0 \text{ vol dir } C_2 \\
1 \text{ vol dir } C_1
\end{cases}
$$
$$
P(t|x,w) =
\begin{rcases}\begin{cases}
g(w^T x) &\text{ si } t = 1 (x \in C_1)\\
1 - g(w^T x) &\text{ si } t = 0 (x \in C_2)
\end{cases}
\end{rcases}
= g(w^T x)^t · [1 - g(w^T x)]^{1 - t}
$$
\begin{flalign*}
	& -l(w) = -\ln \mathcal{L}(w) = -\ln P(\mathcal{D}|w) = 
	-\ln \prod_{n=1}^N P(t_n| x_n, w) \underbrace{=}_{\text{modelem}} \\
	& 5-\ln \prod_{n=1}^N g(w^T x_n)^{t_n} · [1 - g(w^T x_n)]^{1 - t_n} = \\
	& \sum_{n=1}^N \{ t_n \ln g(w^T x_n) + (1 - t_n) \ln(1 - g(w^T x_n)) \} \underbrace{=}_{\text{Defineixo } y_n := g(w^T x_n)} \\
	& -\sum_{n=1}^N \{ t_n \ln y_n + (1 - t_n) \ln (1 - y_n) \}
\end{flalign*}
$$
g'(z) = g(z)[1 - g(z)] \implies \\
g'(w^T x_n) = g(w^T x_n)·[1 - g(w^T x_n)] = y_n (1 - y_n)
$$
\begin{flalign*}
&\textbullet -\sum_{n=1}^N \left\{ \frac{t_n}{y_n}·g'(w^T x_n)·x_n - \frac{1 - t_n}{1 - y_n}·g'(w^T x_n) \right\} =\\
& -\sum_{n=1}^N t_n(1 - y_n)x_n - (1 - t_n)y_n x_n = \\
&- \sum_{n=1}^N \{ t_nx_n - \cancel{t_n y_n x_n} - y_n x_n + \cancel{t_n y_n x_n} \} = \\
& - \sum_{n=1}^N (t_n x_n - y_n x_n) = -\sum_{n=1}^{N} (t_n - y_n) x_n =
\boxed{\sum_{n=1}^N (y_n - t_n)x_n }
\end{flalign*}

\section{El mètode de Newton-Raphsum (NR)}
Va molt bé quan la funció a optimitzar és convexa. Per minimitzar $E(w)$ convexa, es fa el pas següent:
$$
\boxed{w^{\text{(nou)}} := w^{\text{(vell)}} - H^{-1} (w^{\text{(vell)}}) · \nabla E(w^{\text{(vell)}})}
$$
$$
\nabla E(w) =
\begin{pmatrix}
\frac{\partial E}{\partial w_0} \\
\frac{\partial E}{\partial w_1} \\
\vdots\\
\frac{\partial E}{\partial w_d}
\end{pmatrix}
\qquad
H(w) = \left( \frac{\partial^2 E}{\partial w_i \partial w_j} \right)
$$

\begin{lstlisting}[escapeinside={(*}{*)},frame=single]
inicialitzem w(0)
	i := 0
repetir
	w(i + 1) := w(i) - (*$H^{-1}$*)(w(i))·(*$\nabla$*)E(w(i))
	i = i + 1
fins que convergeix
\end{lstlisting}

\subsection{Aplicació af = E(w)}

$$
\nabla E = \sum_{n=1}^N (y_n - t_n) x_n = X^T(y-t)
$$
$$
X_{N\times (d+1)} =
\begin{pmatrix}
\longleftarrow x_1 \longrightarrow \\
\longleftarrow x_2 \longrightarrow \\
\vdots
\longleftarrow x_N \longrightarrow
\end{pmatrix}
\qquad
H = X^T R X
$$
$$
R := diag(y_1(1 - y_1), ..., y_N(1 - y_N))
$$


\subsection{L'algoritme IRLS}
\begin{lstlisting}[escapeinside={(*}{*)},frame=single]
inicialitzar: 
	w(0)(*$_0$*) := (*$\ln \frac{P(C_1)}{P(C_2)}$*)
	w(0)(*$_{1...d}$*) := 0
	i := 0
repetir
	(*$y_n := g(w^T(i)x_n),\ n=1,...,N$*)
	R := diag((*$y_1(1 - y_1),...,y_N(1 - y_N)$*))
	z := (*Xw(i) - $R^{-1}$*)(y - t)
	w(i + 1) := (*$(X^T R X)^{-1} X^T R z$*)
	i := i + 1
fins que convergeix
	(*$R^{-1} = \left( \frac{1}{y_1(1 - y_1)},..., \frac{1}{y_N(1 - y_N)} \right)$*)

Deviance
	D := -2*E(w)
	
Null deviance
	ND := -2*E(w(0))
\end{lstlisting}

% M'he quedat aqui
\begin{algorithmic}[1]
	\State inicialitzar
	\State $w(0)_0 := \ln \frac{P(C_1)}{P(C_2)}$
	\State $w(0)_{1...d} := 0$
\end{algorithmic}

\textbf{AIC} D + 2d
Regressió Logística $\rightarrow$ \texttt{RegLog}

\section{Interpretació de \texttt{RegLog}(I)}
$$
y(x, w) = g(w^T x) = P(C_1|x)
$$
$$
g(z) = \frac{1}{1 + e^{-z}} \qquad g^{-1}(z) =  \ln\left(\frac{z}{1 - z}\right)
$$
$$
w^T x = g^{-1} (P|C_1|x) = \ln\left( \frac{P(C_1|x)}{1 - P(C_1|x)} \right)
$$

\section{Interpretació de \texttt{RegLog}(III)}
Fixem un punt $x^* \in \mathbb{R}^d$ que volem predir (w és el vector solució).
$$
\frac{P(C_1|x^*)}{P(C_2|x^*)} = \exp(w^T x^*)
\qquad 1_i := (\underbrace{0,0,...}_{i - 1},1,...,0)^T
$$
$$
\frac{\frac{P(C_1|x^* + 1_i)}{P(C_2|x^* + 1_i)}}{\frac{P(C_1|x^*)}{P(C_2|x^*)}} =
\exp(w^T(\cancel{x^*} + 1_i) - \cancel{w^T x^*}) = 
\exp(w^T 1_i) = \exp(w_i)
$$
\documentclass[a4paper]{article}
%\usepackage[margin=2cm]{geometry}

\usepackage{amsmath}
\usepackage{amssymb}
\usepackage{amsfonts}
\usepackage{mathtools}
\usepackage[catalan]{babel} % Language 
\usepackage{fontspec}
\usepackage{graphicx}
\usepackage[makeroom]{cancel}
\usepackage{float}
\usepackage{enumerate}
\usepackage{pgfplots}
\usepackage[hidelinks]{hyperref}
\usepackage{listings}
\usepackage{tikz}
\usepackage{ifthen}
\usepackage{array}

\pgfplotsset{compat=1.13}

\tikzset{%
	every neuron/.style={
		circle,
		draw,
		minimum size=1cm
	},
	neuron missing/.style={
		draw=none, 
		scale=4,
		text height=0.333cm,
		execute at begin node=\color{black}$\vdots$
	},
}

\setlength{\parindent}{0pt}
\setlength{\parskip}{0.2cm}

\title{Tema 7: Xarxes neuronals (artificials)}
\author{Joan Marcè Igual}


\begin{document}
\maketitle

\section{Introducció}
Models $y(x) = g(w^T \phi (x)),\ x \in \mathbb{R}^d, w \in \mathbb{R}^{d + 1}$.
$$
\phi(x) = 
\begin{pmatrix}
\phi_0(x) = 1 \\
\phi_1(x)\\
\vdots\\
\phi_M(x)
\end{pmatrix}
\quad 
g : f_n \ g:\mathbb{R} \rightarrow \mathbb{R} \text{(inversa global)}; 
\quad \phi : \mathbb{R}^d \rightarrow \mathbb{R}^M
$$
$$
X_{N \times d+ 1} \xrightarrow{\phi} \Phi_{N \times (M + 1)}
$$

La manera usual d'obtenir models \emph{no lineals} és establir funcions de base \emph{parametritzades}. La presència d'aquests paràmetres no lineals fa que les $\phi$ no es puguin pre-calcular.

En Xarxes Neuronals, la tria d'aquestes funcions de base es fa de la següent manera:
$$
\phi_i(x) := \phi (\psi(x, v_i))
$$
\begin{align*}
	&v_i & \text{és un vector de paràmetres (no lineals)} \\
	&\psi : \mathbb{R}^d \times \mathbb{R}^d \rightarrow \mathbb{R}
	& \text{és una funció de combinació} \\
	&\phi : \mathbb{R} \rightarrow (a,b) &
	\text{és una funció d'activació}
\end{align*}

$\psi$ calcula una similitud entre dos vectors: $x_1 v_i$.
$\phi$ determina el valor final d'activació de $\phi_i$.

\textbf{Definició}: diem que una funció $\phi \mathbb{R} \rightarrow (a,b)$ és \emph{sigmoidal} si:
\begin{enumerate}
	\item $\lim\limits_{z \rightarrow - \infty} \phi(b) = a$
	\item $\lim\limits_{z \rightarrow + \infty} \phi(z) = b$
	\item $\phi'(z) > 0,\ \forall z \qquad \phi'(z) < 0,\ \forall z$ i $\phi'$ té forma de campana.
\end{enumerate}

\textbf{Exemples}:

\begin{enumerate}
	\item La funció logística $\phi(z) = \frac{1}{1 + e^{-z}} \in (0, 1)$
	\begin{figure}[H]
		\centering
		\begin{tikzpicture}
		\begin{axis}[
		xmin=-7,xmax=7,ymin=0,ymax=1,
		domain=-10:10,
		axis x line=center,
		axis y line=center
		]
		\addplot+[mark=none] {1/(1 + exp(-x))};
		\end{axis}
		\end{tikzpicture}
		\caption{Gràfic funció logística}
		\label{fig:logistic}
	\end{figure}

	\item La tangent hiperbòlica $\phi(z) = \frac{e^z - e^{-z}}{e^z + e^{-z}} \in (0,1)$.
	
	\begin{figure}[H]
		\centering
		\begin{tikzpicture}
		\begin{axis}[
		xmin=-5,xmax=5,ymin=-1.05,ymax=1.05,
		domain=-5:5,
		axis x line=center,
		axis y line=center
		]
		\addplot+[mark=none] {(exp(x) - exp(-x))/(exp(x) + exp(-x))};
		\end{axis}
		\end{tikzpicture}
		\caption{Gràfic funció inversa hiperbòlica}
		\label{fig:inv_hiperbol}
	\end{figure}
\end{enumerate}

$$
\implies y(x) = g(w^T \phi(x)) = g\left(\sum_{i=0}^M w_i \phi_i(x)\right) = g \left( \sum_{i=0}^M w_i \phi(v_i^T x + v_{i0}) \right)
$$
\begin{itemize}
	\item $w_i$ són paràmetres lineals
	\item $v_i$ i $v_{i0}$ són paràmetres no lineals
\end{itemize}

% FIGURA 1

Si es dediquen esforços a optimitzar una funció que té molts mínims locals es pot arribar a un mínim que no sigui un mínim absolut. Això dificulta molt la feina d'optimització. Tot i així l'error al que s'acaba arribant al final resulta ser només un error amb les dades d'entrenament. 

Per tant, amb les dades més genèriques l'error també serà menor. Si es minimitzessin al màxim les dades d'entrenament llavors amb les altres potser no s'aconsegueix un mínim.

% FIGURA 2

A cada $ \phi_i(x) $ se li diu \emph{neurona}.

\begin{figure}[H]
	\centering
	\begin{tikzpicture}[x=1.5cm, y=1.5cm, >=stealth]
	
	\foreach \m/\l [count=\y] in {1,2,3,missing,4}
	\node [every neuron/.try, neuron \m/.try] (input-\m) at (0,2.5-\y) {};
	
	\foreach \m [count=\y] in {1,missing,2}
	\node [every neuron/.try, neuron \m/.try ] (hidden-\m) at (2,2-\y*1.25) {\ifthenelse{\equal{\m}{missing}}{}{$S(\Sigma)$}};
	
	\foreach \m [count=\y] in {1,missing,2}
	\node [every neuron/.try, neuron \m/.try ] (output-\m) at (4,1.5-\y) {\ifthenelse{\equal{\m}{missing}}{}{$g(\Sigma)$}};
	
	\foreach \l [count=\i] in {1,2,3,d}
	\draw [<-] (input-\i) -- ++(-1,0)
	node [above, midway] {$I_\l$};
	
	\foreach \l [count=\i] in {1,M}
	\node [above] at (hidden-\i.north) {$\phi_\l$};
	
	\foreach \l [count=\i] in {1,n}
	\draw [->] (output-\i) -- ++(1,0)
	node [above, midway] {$O_\l$};
	
	\foreach \i in {1,...,4}
	\foreach \j in {1,...,2}
	\draw [->] (input-\i) -- (hidden-\j);
	
	\foreach \i in {1,...,2}
	\foreach \j in {1,...,2}
	\draw [->] (hidden-\i) -- (output-\j) node [midway,above,sloped] {$w_{\i\j}$};
	
	\foreach \l [count=\x from 0] in {Input (abstract), Hidden, Ouput}
	\node [align=center, above] at (\x*2,2) {\l \\ layer};
	
	\end{tikzpicture}
\end{figure}

\begin{itemize}
	\item La xarxa treballa "endavant" quan ho fa en el sentit de les fletxes $\rightarrow$.
	\item La capa de \emph{sortida} és la darrera capa endavant
	\item Les capes \emph{ocultes} són totes excepte la de sortida (les seves \emph{neurones} també es diuen \emph{ocultes}).
	\item No hi ha capa "d'entrada".
\end{itemize}

A vegades tenim múltiples capes ocultes

\begin{figure}[H]
	\centering
	\begin{tikzpicture}[x=1.5cm, y=1.5cm, >=stealth]
		\foreach \m [count=\y] in {1,2,missing,3} 
		\node [every neuron/.try, neuron \m/.try](input-\m) at (0,2.5 - \y*1.25){};
	
		\foreach \m [count=\y] in {1,2,missing,3} 
		\node [every neuron/.try, neuron \m/.try](hidden-1-\m) at (2,2.5 - \y*1.25){};
		
		\foreach \m [count=\y] in {1,2,missing,3} 
		\node [every neuron/.try, neuron \m/.try](hidden-2-\m) at (4,2.5 - \y*1.25){};
		
		\node [every neuron/.try](output) at(6,0){};
		
		\foreach \i in {1,...,3}
		\foreach \j in {1,...,3}
		{ 	
			\draw [<-] (input-\i) -- ++(-1,0) node [above, midway] {$I_\i$};
			\draw[->](input-\i) -- (hidden-1-\j); 
			\draw[->](hidden-1-\i) -- (hidden-2-\j);
			\draw[->](hidden-2-\i) -- (output); 
		}
	
		\foreach \l [count=\x from 0] in {Input (abstract), Hidden 1, Hidden 2, Ouput}
		\node [align=center, above] at (\x*2,2) {\l \\ layer};
	\end{tikzpicture}
\end{figure}

$$
y(x) = g\left(\sum_i w_i\phi_i(x)\right) =
g\left(\sum_i w_i \phi (v_i^T x) \right) =
g\left( \sum_i w_i \phi \left(\sum_j v_{ij} x_j\right)  \right)
$$
En aquest cas $x_j \rightarrow \gamma_i (x) = \phi (z_j^T x + z_{j0})$. Per tant:
$$
y(x) = g\left( \sum_i w_i \phi \left( \sum_j w_{ij} \phi \left( z_{jn} x_n + x_{n0} \right) + v_{j0} \right) + w_0 \right)
$$

\section{La funció g}

\begin{table}[H]
	\centering
	\setlength\extrarowheight{15pt}
	\begin{tabular}{p{2.2cm}|p{2.3cm}p{3cm}p{2.8cm}}
		& Regressió & \parbox{2cm}{Classificació \\ ($K=2$)} & \parbox{2cm}{Classificació \\ ($K > 2$)} \\
		\hline
		Funció g & g = Identitat & logística & softmax \\
		Funció d'error & error quadràtic & entropia creuada & entropia creuada generalitzada \\
		Interpretació de la sortida & el target K-èsim (m neurones de sortida) & \parbox{3cm}{$y(x) = P(w_1 |X)$ \\ $\implies$ \\ $P(w_2|X) = 1 - g(x)$} (una neurona de sortida) &  \parbox{2.8cm}{$y_k(x) = P(w_k|X)$ \\ (K neurones de sortida)}
	\end{tabular}
\end{table}

\end{document}
\documentclass[a4paper]{article}

\usepackage{amsmath}
\usepackage{amssymb}
\usepackage{amsfonts}
\usepackage{mathtools}
\usepackage[catalan]{babel} % Language 
\usepackage{fontspec}
\usepackage[makeroom]{cancel}

\setlength{\parindent}{0pt}
\setlength{\parskip}{0.2cm}

\title{Tema 8: Màquines de vectors suport (Introducció als mètodes kernel)}
\author{Joan Marcè i Igual}

\begin{document}
\section{Introducció}

Estem treballant amb models de la forma:
$$
y(x) = g(w^T\phi(x) + w_0),\ x \in \mathbb{R}^d, w \in \mathbb{R}^M, w_0 \in \mathbb{R}^d
$$
$$
\phi(x) =
\begin{pmatrix}
\phi_1(x) \\
\phi_2(x) \\
\vdots \\
\phi_M(x)
\end{pmatrix}
$$

En xarxes neuronals es va dir que $\phi_i(x) = \phi(\varphi(x, v_i))$. 

Ara calculem un model lineal en l'espai de representació generat per les $\phi_i(x)$ (\textbf{feature space} = espai de característiques). El que fan els mètodes no lineals és buscar l'espai d'estadístiques adequat per a un problema en concret.

Una manera diferent de treballar és estudiar l'expressió:
\begin{align*}
	\phi(x)^T\phi(y) \quad x,y \in \mathbb{R}^d \\
	\phi: \text{feature map} 
\end{align*}

\textbf{Definim:} una funció de kernel $k: \mathbb{R}^d \times \mathbb{R}^d \rightarrow \mathbb{R}$ com:
$$
k(x,y) := \phi(x)^T\phi(y)
$$

\begin{itemize}
	\item Es poden estudiar les propietats de $k$ sense calcular (ni tan sols conèixer) $\phi$ de manera explícita.
	\item Es pot obtenir un algorisme d'aprenentatge \textbf{no lineal} partir d'un de lineal mitjançant l'ús de funcions de kernel (\emph{kernelitzar}).
	$$
	x^Ty \rightarrow \underbrace{\phi(x)^T\phi(y)}_{\mathclap{\text{$\phi$ no lineal, el resultat és un algorisme no lineal}}} = k(x,y)
	$$
\end{itemize}

\section{Com construir funcions de kernel}
Si tenim un 
$$
\phi(x) = 
\begin{pmatrix}
\phi_1(x)\\
\phi_2(x) \\
\vdots \\
\phi_n(x)
\end{pmatrix}
$$
$$
\implies k(x,y) = \phi^T(x)\phi(x) = \sum_{i=1}^M \phi_i(x)\phi_i(y)
$$

\textbf{Exemple} $(x^Ty)^3 \ ,x,y \in \mathbb{R}^3$
$$
x = \begin{pmatrix}
x_1 \\
x_2 \\
x_3
\end{pmatrix};
\quad 
y = \begin{pmatrix}
y_1 \\
y_2 \\
y_3
\end{pmatrix}
$$
\begin{align*}
	& (x^Ty)^3 = (x_1y_y + x_2y_2 + x_3y_3) = \\
	& (x_1y_1)^3 + 3(x_1y_1 + x_2y_2)^2 x_3y_3 + 3(x_1y_1 +
	 x_2y_2)(x_3y_3)^2 + (x_3y_3)^3 = ... \\
	& = 
	\begin{pmatrix}
	x_1^3 \\
	x_2^3 \\
	x_3^3 \\
	\sqrt{6} x_1x_2x_3 \\
	\sqrt{3} x_1^2x_2 \\
	\sqrt{3} x_1^2x_3 \\
	\sqrt{3} x_2^2x_1 \\
	\sqrt{3} x_2^2x_3 \\
	\sqrt{3} x_3^3x_1 \\
	\sqrt{3} x_3^3x_2 \\
	\end{pmatrix}^T 
	\begin{pmatrix}
	y_1^3 \\
	y_2^3 \\
	y_3^3 \\
	\sqrt{6} y_1y_2y_3 \\
	\sqrt{3} y_1^2y_2 \\
	\sqrt{3} y_1^2y_3 \\
	\sqrt{3} y_2^2y_1 \\
	\sqrt{3} y_2^2y_3 \\
	\sqrt{3} y_3^2y_1 \\
	\sqrt{3} y_3^2y_2
	\end{pmatrix}
	= \phi(x)^T\phi(y) \qquad \phi:\mathbb{R}^3 \rightarrow \mathbb{R}^{10}
\end{align*}

% FIGURA 2

$$
(x^Ty + 1)^3 \implies
$$
$\phi$ calcula tots els monomis de les variables (1,...,d) fins a grau 3. (0, 1, 2, 3).

$x^Ty$ \textbf{kernel lineal}

$\phi(x) = x$

$(x^Ty + 1)^2$ kernel quadràtic

$(x^Ty + 1)^3$ kernel cúbic

$(x^Ty + c)^q \quad ,c \ge 0, q \in \mathbb{N}$ kernel polinòmic

\textbf{Teorema} Una funció $k:\mathbb{R}^d \times \mathbb{R}^d \rightarrow \mathbb{R}$ és una \textbf{funció lineal} ($\exists \phi: \mathbb{R}^d \rightarrow \mathbb{R}^D, t.q. \ k(x,y) = \phi^T(x)\phi(y)$) si i només si:
\begin{itemize}
	\item $k$ és simètrica
	$$
		k(x,y) = k(y,x) \quad ,\forall x,y \in \mathbb{R}^d
	$$
	\item $\forall N \in \mathbb{N}$, per tota tria $x_1,...,x_N \ \mathbb{R}^d$, la matriu $K_{i,j} := k(x_i, x_j)$ és semi-definida positiva (PSD)
\end{itemize}

\textbf{Definició:} diem que una matriu $K_{N \times N}$ és PSD si és simètrica i $\forall c \in \mathbb{R}^N, c^T K c \ge 0$. És a dir $\sum_{i=1}^N\sum_{j=1}^N c_i K_{ij} c_j \ge 0$.

\textbf{Demostració:} Si la funció $k$ és un kernel $\implies K$ és PSD.
\begin{align*}
	& \sum_i \sum_j c_i K_{ij} c_j = 
	\sum_i \sum_j c_i \phi^T(x_i) \phi(x_j)c_j = \\
	& \left( \sum_i c_i \phi(x_i) \right)^T
	\left( \sum_j c_j \phi(x_j) \right) = 
	\left|\left| \sum_i c_i \phi(x_i) \right|\right|^2 \ge 0
\end{align*}

\textbf{Propietats}
Si $k$, kernels, $a \in \mathbb{R} \quad a \ge 0$:
\begin{itemize}
	\item $a·k$ kernel
	\item $k_1 + k_2$ kernel
	\item $k_1·k_2$ kernel
	\item p(k) kernel, si p és un polinomi de coeficients no negatius
\end{itemize}

\textbf{Proposició}
$$
k(x,y) = (x^Ty + c)^q \quad \forall c \ge 0, \forall q \in \mathbb{N}
$$

Podem definir funcions de kernel en qualsevol conjunt d'objectes o espai.
$$
k: \mathbb{R}^d \rightarrow \mathbb{R}^d 
\rightarrow \mathbb{R} \implies 
k: X \times X \rightarrow \mathbb{R}
$$
$X$ conjunt d'objectes finit (persones).

Una manera de comparar dos conjunts és comparar quants elements tenen en comú
$$
k(x,y) = | x \cap y|
$$
$$
x,y \subset X
$$
$$
I_X (M) = \begin{cases}
1 & \text{ si } M \in X \\
0 & \text{ si } M \cancel{\in} X
\end{cases}
$$
$$
|x \cap y| = \sum_{u \in X} I_X(u)·I_y(u) = k(x,y)
$$

\end{document}


\end{document}