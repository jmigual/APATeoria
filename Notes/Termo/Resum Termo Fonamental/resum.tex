\documentclass[a4paper]{article}
\usepackage[margin=2cm]{geometry}
\usepackage{fontspec}

\usepackage[catalan]{babel}
\usepackage{amsmath}
\usepackage{amssymb}

\usepackage{csquotes}

\setlength{\parindent}{0pt}
\setlength{\parskip}{1em}

\title{Resum Termodinàmica Fonamental}
\author{Joan Marcè Igual}

\begin{document}
\maketitle

\section{Primer principi de la termodinàmica}

El primer principi de la termodinàmica reflexa la \emph{Llei de conservació de l'energia}. Es formula de la següent manera:

\begin{displayquote}
	\em
	En un sistema adiabàtic (que no hi ha intercanvi de calor amb altres sistemes o el seu entorn) que evoluciona d'un estat ideal $\mathcal{A}$ a un estat final $\mathcal{B}$, el treball realitzat no depèn ni del tipus de treball ni del procés seguit.
\end{displayquote}

Així doncs per un sistema tancat:

$$
\Delta U = Q + W
$$

$$
Q_P + W_{altres} = \Delta H = \int_{T_1}^{T_2} m C_P dT
$$

$$
Q_V + W_{altres} = \Delta H = \int_{T_1}^{T_2} m C_V dT
$$

On:
\begin{description}
	\item[\boldmath $\Delta U$] és la variació d'energia del sistema
	\item[\boldmath $Q$] és la calor intercanviada pel sistema a través de les parets
	\item[\boldmath $W$] és el treball intercanviat pel sistema amb el seu entorn
\end{description}

\section{Segon principi de la termodinàmica}

\begin{displayquote}
	\em
	La quantitat d'entropia de l'univers tendeix a incrementar durant el temps.
\end{displayquote}

$$
dS \ge \frac{\delta Q}{T} \implies
\begin{cases}
\Delta S &= \int \frac{\delta Q}{T} \\
Q_{rev} &= \int T dS \\
Q_{rev,isoter} &= T \Delta S
\end{cases}
$$

$$
\Delta S_{uni} = \Delta S_{sistema} + \Delta S_{entorn} = \Delta S_{sistema} + \frac{\pm Q_0}{T_0} \ge 0
$$

\begin{align*}
	\text{Procés adiabàtic irreversible} &\implies \Delta S_{sist} = (S_2 - S_1) > 0 \\
	\text{Procés adiabàtic reversible}  &\implies \Delta S_{sist} = (S_2 - S_1) = 0 \\
	\text{Proces isentròpic} &\implies S_2 = S_1
\end{align*}


\end{document}